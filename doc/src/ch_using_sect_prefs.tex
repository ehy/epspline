	\IXpkgu{} provides a \IXarg{dialog} window with which
	a set of \IXnewterm{preferences} can be set. This dialog window is
	invoked by selecting the edit menu
	(see \nameref{sssec:edit_menu}), ``Preferences'' item.
	The dialog window uses a \IXnewterm{tab} type interface, and
	each \IXarg{tab} contains related settings.

	Five buttons near the bottom of the \IXarg{dialog} window
	affect the preferences and the window. These are explained
	in the table \nameref{tab:prefs_dialog_buttons}. Note that
	the buttons listed in the table as ``Help,'' ``Apply,'' ``Cancel,''
	and ``OK,'' might have different labels and appear in a
	different order depending on the current platform or theme
	(or translation).
	
	\begin{table}[htbp]
	\centering
	\textsf{
	\begin{tabular}[c]{ | l || p{6.33cm} | }
	\hline
	\textsc{Button}: & \textsc{Action}:\\ \hline
	\hline \hline
	Restore Defaults & set \IXpkg{} defaults in all items\\ \hline
	Restore Configured & set values from stored configuration in all items\\ \hline
	\hline
	Help & show the help viewer window displaying this section\\ \hline
	Apply & put current values in effect, and leave the dialog window showing\\ \hline
	Cancel & dismiss the dialog window, and discard changed values\\ \hline
	OK & put current values in effect, dismiss the dialog window, and save the values in stored configuration\\ \hline
	\end{tabular}
	}%\textsf
	\caption{The Buttons of the Preferences Dialog Window}
	\label{tab:prefs_dialog_buttons}
	\end{table}

	The \IXarg{dialog} window is designed to allow interaction
	with the main window (see \nameref{sec:window_interface})
	while the \IXarg{dialog} window remains present. (The
	system might force the \IXarg{dialog} to remain above
	the main window, but it can be moved aside.) Changes to settings
	can be tried and changed again without dismissing the dialog
	using the ``Apply''
	button. Finally, effect or discard changes with ``OK''
	or ``Cancel.''
	
	Each \IXarg{tab} of the \IXarg{dialog} is discussed in the
	following sections. (The \IXarg{preferences} \IXarg{dialog}
	was added in \IXpkg{} 0.0.4.3, and is sparse. More items
	may be added in future releases.)

		\subsection{Global Preferences}%WX%4||||
		\label{ssec:prefs_global}
		This \IXarg{tab} contains controls for settings that
		apply to \IXpkg{}'s appearance or behavior.
		Miscellaneous items appear first followed by
		\IXarg{interface} \IXarg{color} controls.
		The miscellaneous items are:
		
			\begin{description}
			  \item[Default extension/suffix for exports] The
			  suffix offered in the file name \IXarg{dialog}
			  displayed when the ``Export As'' \IXarg{menu}
			  item is selected to choose a name for the
			  \IXpov{} \IXarg{SDL} form of your work.
			  If this item is set, the leading \emph{dot}
			  (`.') should be included as it will not be
			  added automatically. This allows a different
			  separator character, or none at all, to be used.
			  \item[Draw grid lines on canvas background] By
			  default the \IXarg{drawing area} will have a
			  grid of lines meant to help with alignment, and
			  to resemble \IXarg{graph paper}. If a blank
			  \IXarg{canvas} is preferred, the grid can be
			  disabled with this item.
			\end{description}

			The \IXarg{color} selection controls combine a
			text field, and a `picker' widget that is invoked
			with the button to the right of the text field.
			The picker will be conventional for the current
			platform, and will not be described here. The text
			field will accept a number of common color names in
			English (it's uncertain whether these names are
			subject to translation), or as red/green/blue values
			specified in either CSS-like or HTML-like form.
			For example, \texttt{rgb(255, 31, 15)} or
			\texttt{\#FF1F0F}
			respectively. The \IXarg{color} control items are:
			\begin{description}
			  \item[Choose background grid color] If the
			  background grid is enabled, it will be drawn
			  with the \IXarg{color} selected here.
			  \item[Choose guides color] Horizontal or vertical
			  guide lines can be drawn onto the \IXarg{canvas}
			  from the scales to the left and top of the
			  \IXarg{drawing area} (see \nameref{ssec:guide_lines}).
			  The color of these guide lines is selected here.
			  \item[Choose background color] The underlying
			  color of the
			  \IXarg{drawing area} is selected with this item.
			\end{description}

		\subsection{\dtypovhdr{} Settings}%WX%4||||
		\label{ssec:prefs_povray}
		Naturally, \IXpov{} is used to display a preview of
		the current work (see \nameref{sssec:help_menu}).
		This \IXarg{tab} provides settings that apply to
		the invocation of \IXpov{}.
		The \IXpov{} settings are:

			\begin{description}
			  \item[\IXpov{} executable file] An alternative may
			  be given here. If this text field contains a file system
			  path rather than a name alone, \IXpkg{} will attempt
			  to use the path directly. If the field contains a name
			  only, the behavior differs for Unix-like systems and
			  MS Windows systems. Under Unix, the executable will be
			  searched for in \texttt{\$PATH} from \IXpkg{}'s
			  environment, as would be expected. Under MS Windows,
			  the given name will be substituted in the result of
			  a query for the `.POV' file type application. In fact,
			  under MS Windows, the default value ``povray'' is
			  actually just a placeholder. The `.POV' type should
			  be associated with the executable set by the \IXpov{}
			  installer program. If the \IXpov{} directory contains
			  alternative names, one can be given here. If \IXpov{}
			  was not formally installed (for example, a local build
			  from source) a full path may be given here. \IXpkgu{}
			  does not attempt to search a PATH from the environment
			  under MS Windows.
			  \item[\IXpov{} options (switches)] \IXpov{} command
			  options may be set in this text field. The default
			  is `\texttt{+D +P +Q9 +A}' and at least
			  \texttt{+D} and \texttt{+P} should be left in place.
			  Note that if you are using a \IXpov{} version less
			  than 3.7 under Unix, then you might want to add
			  `\texttt{-visual DirectColor}' to avoid unwanted
			  transparency in the preview window, but \emph{do not}
			  give that option for \IXpov{} for Unix version 3.7 or
			  greater, or under MS Windows. This field should be most
			  useful for \texttt{+W} and \texttt{+H}
			  (width and height).
			\end{description}

	It is expected that more settings will be added to the
	\IXarg{preference} \IXarg{dialog} if further development
	of \IXpkg{} is warranted.
