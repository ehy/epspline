\chapter{Introduction}
\ifMKwx
	\small
\begin{verbatim}
Q: Why is it that the more accuracy you demand from an interpolation
   function, the more expensive it becomes to compute?
A: That's the Law of Spline Demand.
   [A Unix fortune.]
\end{verbatim}
	%\normalfont
	\normalsize
\else
	\epigraphhead[70]{
		\epigraph{
		\textit{Q: Why is it that the more accuracy you demand
			from an interpolation
		   function, the more expensive it becomes to compute?}
		
		\textit{A: That's the Law of Spline Demand.}}%
		{\textsc{A Unix fortune.}}%
	}% \epigraphhead[70]
\fi % \ifMKwx

% Introduction opening section
\section{Purpose}%WX%3|||SoftwareIntroSection|
\IXpkgu{} is a simple, single-purpose program that might
be helpful to users of the \IXpov{} program.
For those not already familiar with \IXpov{},
a brief introduction to \IXpov{} follows in the next section,
and the section following that will introduce \IXpkg.

	%WX%4|Brief Introduction to POV-Ray<R>Brief[ 	][ 	]*Introduction[ 	][ 	]*to[ 	].*POV-Ray||POVRayIntroSection|
	\subsection{Brief Introduction to \dtypov}
\IXpov{} is a \IXnewterm{ray-tracing} program with a long history.
Ray-tracing is a method for rendering
graphical images with a model of optics.
Put simply, the user describes objects in space, a viewing position,
and sources of light. The program `traces' the path of a ray of
light from the view, among the objects in space, and back to the sources
of light. The method is repeated for each pixel of the image that
is being generated, and the pixels are colored according to the
interaction of the light and the objects in space that it encounters,
as seen from the view position defined by the user.
The result of this method can be very impressive, and with its
long history of active development \IXpov{} does it very well.

The user of a ray-tracing program must describe objects in space
and their attributes. For that purpose \IXpov{} provides a
\IXnewterm{scene description language}, or `\IXarg{SDL}.'
The \IXarg{SDL}
allows users to describe objects in numerous ways, all of which
are explained in detail, with examples, in \IXpov{}'s
excellent \href{http://www.povray.org/documentation/}{documentation}
at \url{http://www.povray.org/documentation/}.
Many of the objects that can be defined in the \IXarg{SDL} are easily
edited (i.e., written or composed) by hand, but some types of objects
are more suited to generation by an interactive program that
provides visual editing and feedback. \IXpov{} is not
interactive\footnote{The version of \IXpov{} for the
Microsoft Windows platform includes an interactive text editor and
some extra features, but the actual ray tracer still works
with prepared input non-interactively.},
it works with prepared input files. Fortunately, there are several
third-party programs available for interactively editing
\IXpov{} \IXarg{SDL} objects. Such programs typically
produce \IXarg{include} files with object definitions
that the the user will then refer to and arrange in a main scene file.

\IXpov{} is available cost-free as source code, and
ready-to--run binaries for some platforms, at
\IXpov's \href{http://www.povray.org/}{web site}:
\url{http://www.povray.org/}.
\footnote{Please note that, although the source code is available,
it is \emph{not} re-distributable in the manner allowed
by \emph{free software} or \emph{open source} licenses.
It is important that software licenses be respected.
The \IXpov{} license for personal use is at
\url{http://www.povray.org/povlegal.html}, and for distribution
at \url{http://www.povray.org/distribution-license.html}.}

\begin{figure}[htb]
\centering
\includegraphics[width=\linewidth]{\ImgIntroA}
\caption{An image rendered with \dtypov.}
\label{fig:pov_image_Intro_0}
\end{figure}


	%WX%4|Introducing Epspline<R>Introducing .*Epspline||EpsplineIntroSection|
	\subsection{Introducing \dtypkgu}
Two of the object types that may be defined in \IXpov's
\IXarg{scene description language} (\IXarg{SDL})
are the `prism' and the `lathe.'
The purpose of \IXpkg{} is to provide a graphical,
interactive editor of those objects. The prism and lathe
objects are based on interpolated curves, often called
\IXnewterm{spline} curves. Such curves are defined by
the method of interpolation and \IXnewterm{control points}.
Several sets of control points can be grouped together, and
if done well, can compose complex, and \emph{scalable},
shapes.
The shapes of the characters on this page are, almost certainly,
defined by spline curves in a computer typeface, or `font.'

\begin{figure}[htb]
\centering
\includegraphics[width=\linewidth]{\ImgIntroB}
\caption{\dtybeznu{} spline with linked control points selected.}
\label{fig:bezier_linked_control_points}
\end{figure}

\IXpov{} renders spline curves as three dimensional objects.
Figure~\ref{fig:pov_image_Intro_0} is a sample scene
composed mostly of \IXpov's prism and lathe objects
(there is also a `plane' to provide a background, and of course
light sources and a `camera', which is the point-of--view).
The objects arranged in space
are chisels. The handles and ferrules of the chisels are
\IXpov{} lathes, and the blades and decorative imprints
on the handles are prisms.

The difference between the prism and lathe is that
with the former the curve is extruded in the $y$-direction,
and with the latter it is rotated around the $y$-axis.
Making either of these objects in the \IXpov{} \IXarg{SDL}
is similar: the control points necessary to define the
basic curves are placed, as text, in a description of
the object along with other attributes such as texture.
That can be done by hand in a text editor for a small
number of simple objects, but numerous complex objects
would be difficult with hand editing, and
without graphical feedback.

\begin{figure}[htb]
\centering
\includegraphics[width=\linewidth]{\ImgIntroC}
\caption{\dtypov{} preview of the chisel parts.}
\label{fig:chisel_edit_preview_1}
\end{figure}

\IXpkgu{} lets the user place control points with the
\IXnewterm{mouse} in sequence to create a shape.
Existing shapes can be edited in several ways, and also
duplicated, deleted, or transformed.
Figure~\ref{fig:bezier_linked_control_points}
shows a \IXarg{\dtybezil} spline being edited with the mouse.
The cyan colored square is a \IXnewterm{selected} control
point. The cyan colored circles are control points
associated with the selected point and may be moved along with
it. The red colored circles show control point
positions when they are not selected. The shapes visible
in Figure~\ref{fig:bezier_linked_control_points} were
used for the imprint on the chisel handles in
Figure~\ref{fig:pov_image_Intro_0}. The truth about
the selected control point, the cyan colored square, is
that it is two control points with the same coordinates,
and they belong to neighboring curves (which might be
called segments if it is easier to think of the whole
shape as one curve). This mode of editing should be easier
than simply placing numbers in a text editor.

\IXpkgu{} saves the prism and lathe data in its own type
of file rather than \IXpov{} \IXarg{SDL}.
To use the objects in a \IXpov{} scene they
must be \IXnewterm{exported} to an \IXarg{SDL} file that will be
\IXarg{included} by another \IXarg{SDL} file. While
working with \IXpkg{} the current file's objects can
be previewed. \IXpov{} is invoked for this with a simple
\IXarg{SDL} file that is deleted when \IXpov{} is closed.
Figure~\ref{fig:chisel_edit_preview_1}
shows the sample chisel parts in a \IXpov{} preview. Some
of the shapes seen in the preview are not visible in
the scene, but are used for
\IXnewterm{solid contructive geometry} for features
such as the sharp end of the blade and the tang inserted
in the handle.


\section{Requirements and Status}%WX%3||||

\IXpkgu{} uses a software library called \IXnewterm{wxWidgets}
(formerly called \emph{wxWindows}), which provides the window
interface, and event model. \emph{wxWindows} features
portability across several computer platforms. Currently,
\IXpkg{} can be built for, and has been tested on, the
X Window System with the GTK2 toolkit on several
Unix-like systems,\footnote{
\emph{OpenBSD}, \emph{NetBSD}, \emph{FreeBSD},
\emph{OpenSolaris}, \emph{OpenIndiana}, and
\emph{Debian} and \emph{Ubuntu} \emph{GNU/Linux}.
}
and Microsoft Windows.\footnote{
The \emph{Vista} and \emph{7} releases have been tested.
The binaries are built with the \emph{MinGW} tools, and
the the cost-free \emph{DMC} tools have been used too,
and as of this writing, will still build the code against
\emph{wxWidgets} 2.8.
}
Other platforms that are supported by \emph{wxWidgets} have
not been tested and
more work would certainly be needed to
build \IXpkg{} on those platforms
(excepting other Unix-like systems, which might
need only small changes if \emph{wxWidgets} is supported).
\IXpkgu{} has been
tested with \emph{wxWidgets} versions 2.6, 2.8, and the
development version 2.9.

