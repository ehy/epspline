	Working with \IXpkg{} is working with \IXspline{} curves.
	\IXSpline{} curves are common to many programs of
	several types, such as `drawing' programs and typeface
	editors. Of course, any such program must design a
	means of using an input device, usually a \IXnewterm{mouse},
	to place splines on the drawing area. Programs meant
	for drawing might devise powerful and intuitive mouse
	interactions that focus on a graphical result.
	\IXpkgu{} does not. In \IXpkg{}, placing \IXspline{}
	objects on the drawing area is meant to follow the
	\IXpov{} \href{\URLPOVdocs}{documentation}
	on
	\href{\URLPOVdocsLathe}{\IXlathe{}}
	and
	\href{\URLPOVdocsPrism}{\IXprism{}}
	objects.

	As mentioned in \nameref{ssec:intro_epspline},
	there is an important difference
	in the ways \IXpov{} will render prisms and lathes.
	The \IXprism{} is extended (like an ``extrusion'') along the
	$y$-axis. Therefore, the \IXspline{} object for the \IXprism{}
	is drawn parallel to the $x$-$z$ plane in three-dimensional space.
	The \IXlathe{} is rotated \emph{around} the $y$-axis. So, its
	\IXspline{} object is drawn parallel to the $y$-axis. Put another way,
	when editing with \IXpkg{}, if the object is a \IXprism{}
	then the user's \IXnewterm{view} is parallel to the
	$y$-axis, and perpendicular to the $x$-$z$ plane. If the
	object is a \IXlathe{}, the view is perpendicular to the $y$-axis
	(and $x$-axis, even though the \IXlathe{} is rotated).
	
	Editing with \IXpkg{}, the user may mix \IXprism{} and \IXlathe{}
	objects in the same file, but should remember that the
	\emph{view} of each type is different. When generating
	temporary \IXarg{SDL} for a preview with \IXpov{}, \IXpkg{}
	adds a \IXarg{rotate} transform to the \IXprism{} objects
	so that the preview will be more useful. \IXpkgu{} does
	not rotate objects in \IXarg{SDL} exported for
	inclusion in the user's scene, so the user must add
	rotations as necessary.
	
	The coordinate system of the \IXarg{drawing area} in \IXpkg{}
	is, frankly, simply the most convenient for the programming
	of \IXpkg{}. The graduated scales beside the drawing area
	will make it clear that the vertical axis increases
	in the down direction. This is easily corrected using
	transforms available in the \IXarg{SDL}, and will be discussed
	in the section~\nameref{sec:useful_transforms}.

	Hopefully, the following subsections will be clear enough
	to get started and begin realizing ideas. Where
	this document is insufficient, the \IXpov{}
	\href{\URLPOVdocs}{documentation} should help;
	it is necessary to understand how \IXpov{} handles the
	\IXprism{} and the \IXlathe{} objects.
	Although the following subsections are not written as a
	step-by--step tutorial, it should be possible
	(and it is suggested) to follow along with a running
	instance of \IXpkg{}.

		\subsection{Creating New Objects}%WX%4||||
		\label{ssec:creating_newobj}
		Started without file arguments, \IXpkg{}'s
		main window will open with one empty tab.
		The drawing area (the area with the grid
		of light blue lines) is ready, but note that it
		doesn't respond to a (primary) mouse click alone. To start
		a new \IXspline{}, press and hold the \IXnewterm{shift}
		key and click a suitable point with the primary
		mouse button. For practice, a ``suitable'' point
		might be anywhere. A small red-filled circle should
		appear at the point of the mouse click, and that is
		the first control point. Release the shift key to
		continue placing control points. Another shift+click
		before completing at least one \IXspline{} \IXnewterm{segment}
		will cancel, and the control points placed so far will be
		removed. (Recall from the ``Introduction'' chapter
		that the \IXspline{} objects are really several splines
		taken together as a ``path''; from here on ``segment''
		will refer to one \IXspline{} part of the whole \IXspline{} object.)
		Once at least one segment is complete, another
		shift+click will stop editing the object, and the points
		placed so far will remain.
		It will be clear that at least one segment is complete
		when a curve is drawn between some of the control points.
		
		\begin{figure}[htb!]
		\centering
		\includegraphics[width=\linewidth]{\ImgFirstTwoPt}
		\caption{First two control points starting a cubic spline.}
		\label{fig:first_two_points}
		\end{figure}
		
		Figure~\ref{fig:first_two_points} shows the start of a
		\IXspline{} object before a segment is complete, and
		figure~\ref{fig:first_curve_between} shows the first
		curve appear after two more control points have completed
		the segment. The object in the figures is a \IXnewterm{cubic}
		\IXspline{}, which requires four control points for a segment.
		(The number of control points per segment differs among the
		four available \IXspline{} types.) 
		
		\begin{figure}[htb!]
		\centering
		\includegraphics[width=\linewidth]{\ImgFirstSeg}
		\caption{Curve appears between control points of a cubic spline.}
		\label{fig:first_curve_between}
		\end{figure}

		If shift+click is given at the stage of
		figure~\ref{fig:first_two_points}, the two points will
		be deleted, but at the stage of
		figure~\ref{fig:first_curve_between} the points will
		be retained, and editing mode stopped. Editing
		is continued if the object is selected with a click,
		and shift+click given again.

		\begin{figure}[htb!]
		\centering
		\includegraphics[width=\linewidth]{\ImgUnclosedCubic}
		\caption{Cubic spline, not closed.}
		\label{fig:first_cubic_unclosed}
		\end{figure}

		Figure~\ref{fig:first_cubic_unclosed} shows the \IXspline{}
		object nearly complete. More points have been added,
		and another shift+click
		has terminated editing. \IXpov{} would reject the object
		as it is because the curve is not closed, and \IXpkg{}
		will not export it to \IXarg{SDL}.

		\begin{figure}[htb!]
		\centering
		\includegraphics[width=\linewidth]{\ImgClosedCubic}
		\caption{Cubic spline, closed and complete.}
		\label{fig:first_cubic_closed}
		\end{figure}
		
		In figure~\ref{fig:first_cubic_closed} the object has
		been selected with a click. The selected state state is
		indicated by the dotted rectangle (on some platforms the
		rectangle might be dashed). With the whole object
		selected, a control point at an open end was selected,
		indicated by the cyan colored square, and moved to coincide
		with the matching control point. Now that it
		is properly closed, \IXpov{} will render the object.
		Also note that the two points that closed the curve are
		not the first or the last, but rather the second and
		the next-to--last. With the \IXcubic{} \IXspline{} curve, those
		are the points that must match, and the first and last
		are not drawn through. This differs in the other
		\IXspline{} types. \IXpov{} requires a minimum of six
		points in a \IXcubic{} \IXspline{} (\IXpkg{} will draw a \IXcubic{}
		\IXspline{} with five points if points two and four coincide,
		but \IXpov{} will reject the object).
		
		The
		\IXpov{} \href{\URLPOVdocs}{documentation}
		on the
		\href{\URLPOVdocsLathe}{\IXlathe{}}
		and
		\href{\URLPOVdocsPrism}{\IXprism{}}
		objects will explain in more detail how points
		should be placed for each \IXspline{} type, and other
		characteristics of the types.
		
		The \IXcubic{} type of \IXspline{} has been discussed so far
		because it is the default type for new objects if another
		type has not been selected. To choose among the four \IXspline{}
		types, make sure that no existing object is selected by
		clicking the drawing area, outside of the selection
		rectangle of any selected object, with the primary
		mouse button. Next, click the drawing area with secondary
		button, which will invoke a ``pop-up'' menu. The
		drawing area menu is seen in
		figure~\ref{fig:canvas_popup_menu}.

		\begin{figure}[htb!]
		\centering
		\includegraphics{\ImgCtxMenuA}
		\caption{The drawing area ``pop-up'' menu.}
		\label{fig:canvas_popup_menu}
		\end{figure}

		The first (topmost) \IXarg{menu} item ``Curves''
		names the menu, and also serves to dismiss the menu
		without making a selection.
		The \IXarg{menu} items beginning with ``Set \ldots''
		select the type of object to be created. The first four
		of these offer the \IXspline{} type, and the next two offer the
		\IXpov{} object type: \IXprism{} or \IXlathe{}. The last four
		items on the menu
		are equivalent to the ``File'' \IXarg{menu} items
		with the same labels
		(table~\ref{tab:File_menu}), and are placed on this menu
		for convenience. Note that an existing object can
		be changed between a \IXlathe{} and \IXprism{}, so it's not
		necessary to make that choice initially, but an existing
		object cannot be changed to a different \IXspline{} type, so
		a type should be chosen before starting each object.

		\begin{figure}[htb!]
		\centering
		\includegraphics[width=\linewidth]{\ImgUnclosedQuadratic}
		\caption{Quadratic spline, not closed.}
		\label{fig:first_quadratic_unclosed}
		\vspace{16pt}
		\includegraphics[width=\linewidth]{\ImgClosedQuadratic}
		\caption{Quadratic spline, closed and complete.}
		\label{fig:first_quadratic_closed}
		\end{figure}

		Figures~\ref{fig:first_quadratic_unclosed}~and~\ref{fig:first_quadratic_closed}
		are similar to
		figures~\ref{fig:first_cubic_unclosed}~and~\ref{fig:first_cubic_closed},
		but show a \IXquadratic{} \IXspline{}. In this case the points that
		must match to close the curve are the second and the
		\emph{last} (rather than the second-to--last
		point in the \IXcubic{} type).
		\IXpov{} requires at least five points in the closed
		\IXquadratic{} \IXspline{}.

		\begin{figure}[htb!]
		\centering
		\includegraphics[width=\linewidth]{\ImgUnclosedLinear}
		\caption{Linear spline, not closed.}
		\label{fig:first_linear_unclosed}
		\vspace{16pt}
		\includegraphics[width=\linewidth]{\ImgClosedLinear}
		\caption{Linear spline, closed and complete.}
		\label{fig:first_linear_closed}
		\end{figure}

		The equivalent for the \IXlinear{} \IXspline{} type is seen in
		figures~\ref{fig:first_linear_unclosed}~and~\ref{fig:first_linear_closed}.
		There must be at least three points in the \IXlinear{} \IXspline{}.		

		\begin{figure}[htb!]
		\centering
		\includegraphics[width=\linewidth]{\ImgUnclosedBezier}
		\caption{\dtybezieru{} spline, not closed.}
		\label{fig:first_bezier_unclosed}
		\vspace{16pt}
		\includegraphics[width=\linewidth]{\ImgClosedBezier}
		\caption{\dtybezieru{} spline, closed and complete.}
		\label{fig:first_bezier_closed}
		\end{figure}

		The \IXbezn{} \IXspline{} object must be closed too, of course,
		as seen in
		figures~\ref{fig:first_bezier_unclosed}~and~\ref{fig:first_bezier_closed}.
		\IXpov{} will render a \IXbezn{} \IXspline{} object with only
		four points, which is one complete segment, if the first
		and last points coincide.
		The figures show a \IXbezn{} object
		with two segments.

			\subsubsection{The Points of the \dtybezieru{} Type}%WX%5||||
			\label{sssec:editing_points_bezier}
			Quoting the start of section
			\ref{sec:creat_editing},
			``In \IXpkg{}, placing \IXspline{}
			objects on the drawing area is meant to follow the
			\IXpov{} \href{\URLPOVdocs}{documentation}
			on
			\href{\URLPOVdocsLathe}{\IXlathe{}}
			and
			\href{\URLPOVdocsPrism}{\IXprism{}}
			objects.''
			That means that a point must be placed on the canvas
			in the same manner that a \IXnewterm{vector} would
			be added to a \IXprism{} or \IXlathe{} when
			editing \IXarg{SDL},
			with the same requirements of order and position.
			\IXpkgu{} does not add missing points or
			automatically close curves; but, a little help
			is provided when the graphical interface makes
			it necessary: when two points are clearly meant
			to coincide, \IXpkg{} may make them do so, because
			such precise placement with a mouse is very
			difficult. In general, a curve or \IXnewterm{sub-curve}
			must be closed with the user's knowledge of
			which points must coincide.
			
			The \IXbezn{} \IXspline{} type is discussed here in more detail
			because:
			
			\begin{itemize}
				\item It differs from the other types more
						than they differ from each other.
				\item Regardless of the difference, mastering
						the \IXbezn{} might make mastering the
						other types easier.
				\item The \IXbezn{} might be the most
						useful type [in the author's opinion
						that is so].
			\end{itemize}
			
			To begin a new object, and to stop editing it,
			is accomplished the same way for all the \IXspline{} types.
			(If that was not made
			clear above, look forward to the
			\nameref{sssec:new_obj_summary} section.)
			Once a \IXbezn{} \IXspline{} object has been started
			it \emph{must} be remembered that that the whole
			curve is composed of segments, that each segment
			is composed of four control points, that the
			segments must be joined so that the last (fourth) point
			of the previous segment \IXarg{coincides} with
			the first point of the next segment,
			and that the whole curve must be closed by
			placing the last point of the last segment
			\IXarg{coincident} with the first point
			of the first segment of the curve.
			(A \IXbezn{} \IXspline{} object
			may consist of a single segment, in which case
			the first and last points of this one segment
			must coincide.) Therefore, after first starting
			a \IXbezn{} curve, place the second, third and
			fourth points where they make sense, but do not
			move the mouse after placing the fourth point.
			Click again at the position of the fourth point
			to begin the next segment, and its
			first point will \IXarg{coincide} with the
			last of the previous segment, as it should.
			Finally, close the curve by placing a segment's
			last point just at the point that started the curve.
			
			Although \IXpkg{} does little to enforce these
			requirements, \IXpov{} will enforce them.
			\IXpkgu{} does provide a little help: if the
			fourth and first points of two segments in
			sequence were placed offset by a pixel or two,
			\IXpkg{} will join them, so a slight movement
			of the mouse before placing the first point
			of a new segment is not an irrecoverable error.
			If two points that should coincide are separated
			by more than a couple of pixels,
			\IXpkg{} will not join them, but they can be brought
			together with subsequent editing, i.e, after the
			initial creation of the object has been
			finished with another shift-key+primary-click.
			
			Some trial and error practice should make the
			points of this discussion more clear.
			(The ``Undo'' facility will always be very
			helpful.) After a few successful \IXbezn{}
			creations, try the other \IXspline{} types and discover
			how they differ.


			\subsubsection{New Object Summary}%WX%5||||
			\label{sssec:new_obj_summary}
			As a summary of the creation of a new \IXprism{}
			or \IXlathe{}, consider this step-by--step list:
			
			\begin{enumerate}
				\item If any current object is \IXarg{selected}
						(a dashed or dotted box is shown and
						control points are shown with circles),
						\IXarg{deselect} it by clicking the
						drawing area outside the selected object's
						bounding box.
				\item Set the type of \IXspline{}
						to be created by clicking the
						drawing area with the secondary \IXarg{mouse}
						button and choosing from the \IXarg{menu},
						as shown in
						figure~\ref{fig:canvas_popup_menu}.
						Do not forget this step because the
						\IXspline{} type of an existing
						object cannot be changed.
				\item Optionally, choose whether the new object
						will be a \IXprism{} or \IXlathe{}
						from the drawing area \IXarg{menu}.
						This can be changed for an existing
						object, so it is not necessary initially.
				\item Enter editing mode:
						with the shift-key depressed, click the
						canvas with
						the primary mouse button at a suitable
						place for the first control point, and
						a circle representing the point should appear.
				\item Release the shift-key before clicking
						the \IXarg{canvas} again, or the object
						will be cancelled. Of course, the object
						may be cancelled intentionally.
				\item Click to add points, in a manner suitable
						for the type of \IXspline{} being created,
						until the object is complete, or at
						least until a curve segment is drawn between
						points (so that the object will not
						be cancelled and will remain for further
						editing).
				\item Leave editing mode with a primary
						mouse button click with the shift
						key depressed.
			\end{enumerate}


		\subsection{Continued Editing}%WX%4||||
		\label{ssec:cont_editing}
		Typically, an idea
		of a form will suggest where points should be
		placed, but the curve that appears is less than
		ideal. In fact, placing
		\IXarg{points} with mouse clicks, as described
		so far, is just a rough start. If an idea of an
		object for a \IXpov{} \IXarg{scene} can be
		implemented as a \IXpov{} \IXprism{} or
		\IXlathe{}, then \IXpkg{} might help
		\emph{develop} the object. That is, making
		any but the simplest objects will be a process
		of changes and tuning. In this process, the
		interactive graphical interface should be
		an advantage. The use of the facilities
		provided by \IXpkg{} are discussed below.
		
			\subsubsection{Object Properties}%WX%5||||
			\label{sssec:object_props}
			To this point prisms and lathes have been
			referred to repeatedly, but there has been
			little, if any, mention of their properties.
			Moreover, it was stated that although an
			object's spline-type cannot be changed after
			it's created, it can be changed to a \IXprism{}
			or \IXlathe{}.
			
			The properties of objects in \IXpkg{}
			are changed in a \IXnewterm{dialog} window,
			``Spline Properties,'' which is invoked
			through an object \IXarg{menu}. This menu,
			seen in figure~\ref{fig:object_popup_menu},
			will `pop up' in response to a secondary
			button click on a selected object.
			(If no object is selected, the secondary
			button click will produce the drawing area
			menu.)
			
			\begin{figure}[htb!]
			\centering
			\includegraphics{\ImgCtxMenuB}
			\caption{The object ``pop-up'' menu.}
			\label{fig:object_popup_menu}
			\end{figure}

			The object menu has four items (after the label,
			``Selection''). The last three items are
			equivalent to similarly named items on the
			edit \IXarg{menu} (see \nameref{sssec:edit_menu}).
			The first item, as its name suggests, invokes
			the ``Spline Properties'' dialog, shown in
			figure~\ref{fig:props_dlg}.

			\begin{figure}[htb!]
			\centering
			\includegraphics{\ImgPropDlgA}
			\caption{The properties window.}
			\label{fig:props_dlg}
			\end{figure}
			
			The ``Spline Properties'' dialog fields are as follows:
			\begin{description}
			  \item[(label)] A label stating the type of \IXspline{}.
			  \item[Object Name] A text entry field: a name
				for the object, suitable for \IXpov{} \IXarg{SDL},
				should be entered here before export.
			  \item[Render Type] A selection of rendering options:
				as \IXlathe{}, as \IXprism{}, or ``Undefined,''
				which means that it will not
				be rendered; this can be used to hide an object in
				a \IXpov{} \IXarg{scene} while keeping it in the
				\IXpkg{} file. This ``Undefined'' type must be given
				a name too, because it is exported to SDL as a \IXpov{}
				array (which might be useful).
			  \item[Sweep Type] This applies to prisms. The sweep
				type is described well in the \IXpov{}
				\href{\URLPOVdocsPrism}{\IXprism{} documentation}.
				Note that presently a conic sweep \IXprism{} is best
				if moved so that its center occurs at the origin
				of the \IXpkg{} drawing area.
			  \item[Use Sturm, Open] These are optional. Sturm selects
				a more accurate but slower algorithm; unless you
				know that you need this, you probably don't.
				Open applies to the \IXprism{}: the flat faces will not
				be rendered, leaving an open outline.
			  \item[Start, End Sweep] This applies to prisms:
				roughly, the offset and height of the \IXprism{}.
			  \item[Texture, Interior, Transform] Text entry
				fields that can accept a name for the attribute
				that must be
				`\verb!#declare!'ed
				before the exported \IXarg{SDL} is included
				in a scene. These fields appear with a default
				string (until changed) that is not applied to
				the object --- the default is only a place-holder.
				These attributes can probably be applied more
				flexibly in the using SDL.				
			  \item[Predefined Identifiers] This text entry field
				can accept more than one line. It is poorly named
				because it is not restricted to identifiers;
				for example, a comment may be added here.
				The most important point about this field is
				that it is not checked at all by \IXpkg{} ---
				the user must be certain that anything added here
				is valid SDL. The contents will be added near the
				end of the object definition.
			  \item[(buttons)] Dismiss the dialog window, either
				cancelling or effecting any changes that were made.
			\end{description}
			Obviously, some of these properties should be set
			before export to \IXarg{SDL}.

			\subsubsection{\dtypkgu{} Transforms and Copies}%WX%5||||
			\label{sssec:epspline_transforms}
			The transforms provided by \IXpov{} \IXarg{SDL}
			will be necessary in any any scene, and some
			that pertain to \IXpkg{} output are
			discussed in \nameref{sec:useful_transforms}.
			\IXpkgu{} provides some interactive transforms
			that should be useful while developing objects.
			These are applied directly to the control \IXarg{point}
			values rather than as exported SDL statements.
			
			Objects may be moved. Click an object to select it,
			and then \IXnewterm{drag} it with the mouse, and
			release it in the desired position. Single-pixel
			movements can be made with the arrow keys. Larger
			movements can be made with the page up and page down
			keys; if the control (or command) key is down then
			the page-key movements will be horizontal. Note that
			to move the whole object with keys, make sure that
			no control point is selected, or the point will be
			moved.
			
			Objects may be scaled, sheared, or rotated.
			To enter the mode in which these transforms are
			done, first select the object, then hold the
			shift \IXarg{key} down and click with the secondary
			mouse button. An object selected for these
			transforms is seen in figure~\ref{fig:sel_for_trans_a}.
			To leave this mode, another secondary click, but
			\emph{without} the shift key down, will return
			the object to the simple selected state.
			Repeated shift-key+secondary-clicks
			cycle through the scale, shear, and rotate modes,
			in order.

			\begin{figure}[htb!]
			\centering
			\includegraphics[width=\linewidth]{\ImgSelTransformA}
			\caption{Selected object in transform mode.}
			\label{fig:sel_for_trans_a}
			\end{figure}
			
			In transform mode small squares are
			shown just inside the object's bounding box, at the
			corners and at the middle of the sides. These
			squares are \IXnewterm{handles} for manipulating
			the object: \IXarg{drag} the handles to effect the
			transform.
			
			In figure~\ref{fig:sel_for_trans_a}
			two small concentric circles can be seen at the
			middle of the object. These circles are shown for
			shearing and rotation, and may be dragged with the
			mouse to a new location.
			The transform will be centered at the circles.
			They are not shown in the scaling mode, which uses no
			center as such.
			
			For each of the
			scale, shear, and rotate modes the shape of the
			mouse pointer will change. The pointer shapes
			should be suggestive of the mode (in some
			\IXarg{GTK} themes of the last several years the
			pointer shapes no longer seem to make sense, but the
			user should still form an association between the
			pointer shapes and the modes).
			
			Each transform, effected by dragging a handle with the
			mouse, is modified by one or more keys.	If the shift
			key is held down, the shear and rotate transforms
			will be constrained to a 15 degree angle (presently,
			this angle can only be changed as a compile-time
			macro, but may be made configurable
			in a future release if \IXpkg{} warrants more work).
			The scale mode transform, with the shift key, will become
			instead a \IXnewterm{flip} of the object horizontally,
			vertically, or diagonally according to the handle
			used. Also with the scale transform, if the `alt'
			key is held the transform will be directly proportional
			when using the upper-left or lower-right handle-squares,
			or inversely proportional with upper-right or
			lower left handles. The alt key has no effect on
			scaling with the middle-of--the--sides handles, or on
			the shear and rotate transforms.
			
			A copy of the original, un-scaled object is left
			if the control (or command) key is down
			at the start of any transform. To make a copy that
			does not require a transform, simply make a
			\IXarg{clipboard} copy with the edit \IXarg{menu}
			(see \nameref{sssec:edit_menu}) or \IXarg{tool} bar
			(see \nameref{ssec:tool_bar}) and
			\IXarg{paste} the copy back. Whenever a copy is
			made, remember to make sure that each has a unique
			name (this is not done automatically, at present;
			see \nameref{sssec:object_props}).
			
			\subsubsection{Adding, Moving, and Deleting Points}%WX%5||||
			\label{sssec:add_del_points}
			Points can be added to an existing object
			for two purposes:
			as part of a new discontinuous sub-curve, or as
			part of an existing curve. The former case is
			discussed later in the
			section~\nameref{sssec:discont_objects}.
			
			To add a point to an existing curve, first make sure
			the object is selected. Place the mouse active point
			\emph{on} the curve at the desired position.
			Hold the shift \IXarg{key} down and click the
			primary mouse button. If the mouse pointer was
			properly on the curve, a new point will appear as a
			cyan colored square. If a new point appears as
			a red circle, the mouse active point did not lie on the
			curve, and the point should not be left in place.
			The errant new point can be removed with
			``Undo'' on the edit \IXarg{menu}
			(see \nameref{sssec:edit_menu}) or \IXarg{tool} bar
			(see \nameref{ssec:tool_bar}),
			or the \IXnewterm{control+z} key combination.
			
			A `point' is added to the \IXbezn{} curve in the
			same way, but the actual result is that the current
			segment is divided in two at the mouse point, and
			several necessary points are added. Some effort is
			made to place new \IXnewterm{tangent} control
			points sensibly, but adjustment will be needed.
			
			Moving points should be obvious: select the point so
			that it appears as a cyan colored square and drag it
			with the mouse. Only the \IXbezn{} curve needs more
			explanation. The section
			``\nameref{sssec:editing_points_bezier}'' explained
			that the last and first end-points of neighboring
			segments are (must be) coincident. \IXpkgu{}
			actually maintains an association of four points:
			the last and first \IXarg{tangent} control points
			and the end-points of neighboring segments. When any
			of these is selected the whole group will be
			colored cyan, and the selected point will be square.
			Within this group, dragging a \IXarg{tangent}
			control point will also move the other tangent
			point onto a line that passes through the endpoints;
			this produces a smooth curve through the end-points.
			A tangent point can be moved independently by
			pressing the control (or command) key, making
			an angle at the end-points. The end-points can
			be dragged, and they are always kept together
			as one. Dragging the end-points will also make
			an equivalent movement of the
			tangent points, causing a movement of the
			whole group; the end-points can be moved
			without affecting the tangent points by
			pressing the control (or command) key.
			
			Selected points may be deleted with the delete key
			or the edit \IXarg{menu} ``Delete'' item.
			In the case of the
			\IXbezn{} curve a whole segment is
			deleted, and remaining segments are joined at
			end-points. Adjustment will be needed.

			\subsubsection{Discontinuous Objects}%WX%5||||
			\label{sssec:discont_objects}

			\subsubsection{Guide Lines}%WX%5||||
			\label{sssec:guide_lines}

			\subsubsection{Missing Features}%WX%5||||
			\label{sssec:missing_todo}

