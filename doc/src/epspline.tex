% Epspline Program Documentation: Runtime Help
% Ed Hynan -- Dec 31, 2012

% NOTES:
%	htlatex can be invoked as
%		htlatex epspline.tex "html,index=2,4,sec-filename,charset=UTF-8"
%	and the output is nice, but the output if iso-8859-1, so html must
%	be post-processed with iconv -f iso-8859-1 -t UTF-8 for proper display
%	in browsers; but presently I don't know how well the wxWidgets viewer
%	handles charsets, so leave out ,charset=UTF-8 until further testing.
%

% TIPS, from something used as a starter:
% WARNING!  Do not type any of the following 10 characters except as directed:
%                &   $   #   %   _   {   }   ^   ~   \

%Double quotes are typed like this: ``quoted text''.
%Single quotes are typed like this: `single-quoted text'.
%
%Long dashes are typed as three dash characters---like this.
%
%Emphasized text is typed like this: \emph{this is emphasized}.
%Bold       text is typed like this: \textbf{this is bold}.

%Remember, don't type the 10 special characters (such as dollar sign and
%backslash) except as directed!  The following seven are printed by
%typing a backslash in front of them:  \$  \&  \#  \%  \_  \{  and  \}.
%The manual tells how to make other symbols.


%\documentclass[hypertex]{article}% Your input file must contain these two lines
%\documentclass[hypertex]{report}% Your input file must contain these two lines
%\documentclass[hypertex,11pt,onecolumn,letterpaper]{report}%
% Mar 4 2013: removed hypertex driver from documentclass - had been
% there harmlessly (but probably wrongly) since 2005; but, recent
% hyperref package from texlive is choking on it.
\documentclass[11pt,onecolumn,letterpaper]{report}%
	\usepackage{apphelpdoc}%
\begin{document}

% hide numbers for front matter:
\pagestyle{empty}

%%\ifpdf
% \ifMKwx is defined in apphelpdoc.sty, dependent on buildsel.sty:
% tests if making html for wxWidgets help viewer.
\ifelseWX{
	% htlatex making html 3.2, which is used for wx help, doesn't 
	% handle \foo within \title, \chapter, etc.. -- puts the
	% remainder of line in <p></p> above <title>!
	\title{Epspline: An Editor for POV-Ray Prism and Lathe Objects}
}{
	\title{\dtypkgu: An Editor for \dtypov{} Prism and Lathe Objects}
}% \ifelseWX
\author{\thisauthor}
\date{\docreldate}
\maketitle

% Coded comment follows 'WX%': string of fields separated by '|', 
% as follows:
%	F1== logical level: 0==N.G., 1==book, 2==chapter, 3==section, etc.
%	F2== name, or label; if "" than take left of 'WX%', between { and }
%	F3== 32 bit integer identifier : generated incrementally if empty*
%	F4== name for cpp macro, e.g. IDI_$F4; generated if empty
%	F5== name of output file; if "" then override file<->title map
%	*Note on F3: IDs incr from low numbers, so make explicit #s high

% coded comment for title page
%WX%1|Title Page|999999999|TitlePage|epspline.html

% coded comment for Doc License page
%WoffX%1|Document License|999999998|DocumentLicensePage|
\ifMKhtml
	\clearpage
\else
	\cleardoublepage
\fi
{\parindent 0in
 Copyright \copyright\ \doccpyyear\ \thisauthor.
 Permission is granted to copy, distribute and/or modify this document
 under the terms of the GNU Free Documentation License, Version 1.2
 or any later version published by the Free Software Foundation;
 with no Invariant Sections, no Front-Cover Texts, and no Back-Cover Texts.
 A copy of the license is included within the chapter
 entitled ``Licenses''
 in the section entitled ``GNU Free Documentation License''.}
\ifMKhtml
	\clearpage
\else
	\cleardoublepage
\fi

% unhide numbers for contents:
\pagestyle{plain}
% roman numbers for contents:
\pagenumbering{roman}

\tableofcontents
% coded comment for table of contents
%WX%1|Contents|1000000000|HelpTOC|

\ifMKhtml
	\clearpage
\else
	\cleardoublepage
\fi

% unhide numbers for contents:
\pagestyle{headings}
% arabaic numbers for main:
\pagenumbering{arabic}

% 1st chapter, Introduction
\putintro{Introduction}{chap:introduction}

\ifMKwx
	\small
\begin{verbatim}
Q: Why is it that the more accuracy you demand from an interpolation
   function, the more expensive it becomes to compute?
A: That's the Law of Spline Demand.
\end{verbatim}%
	\hrule
	\textsc{A Unix fortune}
	%\normalfont
	\normalsize
\else
	\epigraphhead[70]{
		\epigraph{%
		\textbf{Q}: Why is it that the more accuracy you demand
			from an interpolation
		   function, the more expensive it becomes to compute?\\
		\textbf{A}: That's the Law of Spline Demand.}%
		{\textsc{A Unix fortune}}%
	}% \epigraphhead[70]
\fi % \ifMKwx

% Introduction opening section
\section{Purpose}%WX%3|||SoftwareIntroSection|
	\label{sec:epspline_purpose}
\IXpkgu{} is a simple, single-purpose program that might
be helpful to users of the \IXpov{} program.
For those not already familiar with \IXpov{},
a brief introduction to \IXpov{} follows in the next section,
and the section following that will introduce \IXpkg.

	%WX%4|Brief Introduction to POV-Ray<R>Brief[ 	][ 	]*Introduction[ 	][ 	]*to[ 	].*POV-Ray||POVRayIntroSection|
	\subsection{Brief Introduction to \dtypov}
	\label{ssec:intro_povray}
\IXpov{} is a \IXnewterm{ray-tracing} program with a long history.
Ray-tracing is a method for rendering
graphical images with a model of optics.
Put simply, the user describes objects in space, a viewing position,
and sources of light. The program `traces' the path of a ray of
light from the view, among the objects in space, and back to the sources
of light. The method is repeated for each pixel of the image that
is being generated, and the pixels are colored according to the
interaction of the light and the objects in space that it encounters,
as seen from the view position defined by the user.
The result of this method can be very impressive, and with its
long history of active development \IXpov{} does it very well.

The user of a ray-tracing program must describe objects in space
and their attributes. For that purpose \IXpov{} provides a
\IXnewterm{scene description language}, or `\IXarg{SDL}.'
The \IXarg{SDL}
allows users to describe objects in numerous ways, all of which
are explained in detail, with examples, in \IXpov{}'s
excellent \href{http://www.povray.org/documentation/}{documentation}
at \url{http://www.povray.org/documentation/}.
Many of the objects that can be defined in the \IXarg{SDL} are easily
edited (i.e., written or composed) by hand, but some types of objects
are more suited to generation by an interactive program that
provides visual editing and feedback. \IXpov{} is not
interactive\footnote{The version of \IXpov{} for the
Microsoft Windows platform includes an interactive text editor and
some extra features, but the actual ray tracer still works
with prepared input non-interactively.},
it works with prepared input files. Fortunately, there are several
third-party programs available for interactively editing
\IXpov{} \IXarg{SDL} objects. Such programs typically
produce \IXarg{include} files with object definitions
that the user will then refer to and arrange in a main scene file.

\IXpov{} is available cost-free as source code, and
ready-to--run binaries for some platforms, at
\IXpov's \href{http://www.povray.org/}{web site}:
\url{http://www.povray.org/}.
\footnote{Please note that, although the source code is available,
it is \emph{not} re-distributable in the manner allowed
by \emph{free software} or \emph{open source} licenses.
It is important that software licenses be respected.
The \IXpov{} license for personal use is at
\url{http://www.povray.org/povlegal.html}, and for distribution
at \url{http://www.povray.org/distribution-license.html}.}

\begin{figure}[htb]
\centering
\includegraphics[width=\linewidth]{\ImgIntroA}
\caption{An image rendered with \dtypov.}
\label{fig:pov_image_Intro_0}
\end{figure}


	%WX%4|Introducing Epspline<R>Introducing .*Epspline||EpsplineIntroSection|
	\subsection{Introducing \dtypkgu}
	\label{ssec:intro_epspline}
Two of the object types that may be defined in \IXpov's
\IXarg{scene description language} (\IXarg{SDL})
are the `\IXprism{}' and the `\IXlathe{}.'
The purpose of \IXpkg{} is to provide a graphical,
interactive editor of those objects. The \IXprism{} and \IXlathe{}
objects are based on interpolated curves, often called
\IXnewterm{spline} curves. Such curves are defined by
the method of interpolation and \IXnewterm{control points}.
Several sets of control points can be grouped together, and
if done well, can compose complex, and \emph{scalable},
shapes.
The shapes of the characters on this page are, almost certainly,
defined by \IXspline{} curves in a computer typeface, or `font.'

\begin{figure}[htb]
\centering
\includegraphics[width=\linewidth]{\ImgIntroB}
\caption{\dtybeznu{} spline with linked control points selected.}
\label{fig:bezier_linked_control_points}
\end{figure}

\IXpov{} renders \IXspline{} curves as three dimensional objects.
Figure~\ref{fig:pov_image_Intro_0} is a sample scene
composed mostly of \IXpov's \IXprism{} and \IXlathe{} objects
(there is also a `plane' to provide a background, and of course
light sources and a `camera', which is the point-of--view).
The objects arranged in space
are chisels. The handles and ferrules of the chisels are
\IXpov{} lathes, and the blades and decorative imprints
on the handles are prisms.

The difference between the \IXprism{} and \IXlathe{} is that
with the former the curve is extruded in the $y$-direction,
and with the latter it is rotated around the $y$-axis.
Making either of these objects in the \IXpov{} \IXarg{SDL}
is similar: the control points necessary to define the
basic curves are placed, as text, in a description of
the object along with other attributes such as texture.
That can be done by hand in a text editor for a small
number of simple objects, but numerous complex objects
would be difficult with hand editing, and
without graphical feedback.

\begin{figure}[htb]
\centering
\includegraphics[width=\linewidth]{\ImgIntroC}
\caption{\dtypov{} preview of the chisel parts.}
\label{fig:chisel_edit_preview_1}
\end{figure}

\IXpkgu{} lets the user place control points with the
\IXnewterm{mouse} in sequence to create a shape.
Existing shapes can be edited in several ways, and also
duplicated, deleted, or transformed.
Figure~\ref{fig:bezier_linked_control_points}
shows a \IXarg{\dtybezil} \IXspline{} being edited with the mouse.
The cyan colored square is a \IXnewterm{selected} control
point. The cyan colored circles are control points
associated with the selected point and may be moved along with
it. The red colored circles show control point
positions when they are not selected. The shapes visible
in figure~\ref{fig:bezier_linked_control_points} were
used for the imprint on the chisel handles in
figure~\ref{fig:pov_image_Intro_0}. The truth about
the selected control point, the cyan colored square, is
that it is two control points with the same coordinates,
and they belong to neighboring curves (which might be
called segments if it is easier to think of the whole
shape as one curve). This mode of editing should be easier
than simply placing numbers in a text editor.

\IXpkgu{} saves the \IXprism{} and \IXlathe{} data in its own type
of file rather than \IXpov{} \IXarg{SDL}.
To use the objects in a \IXpov{} scene they
must be \IXnewterm{exported} to an \IXarg{SDL} file that will be
\IXarg{included} by another \IXarg{SDL} file. While
working with \IXpkg{} the current file's objects can
be previewed. \IXpov{} is invoked for this with a simple
\IXarg{SDL} file that is deleted when \IXpov{} is closed.
Figure~\ref{fig:chisel_edit_preview_1}
shows the sample chisel parts in a \IXpov{} preview. Some
of the shapes seen in the preview are not visible in
the scene, but are used for
\IXnewterm{solid contructive geometry} for features
such as the sharp end of the blade and the tang inserted
in the handle.


\section{Requirements and Status}%WX%3||||
	\label{sec:requirement_status}

\IXpkgu{} uses a software library called \IXnewterm{wxWidgets}
(formerly called \emph{wxWindows}), which provides the window
interface, and event model. \emph{wxWindows} features
portability across several computer platforms. Currently,
\IXpkg{} can be built for, and has been tested on, the
X Window System with the GTK2 toolkit on several
Unix-like systems,\footnote{
\emph{OpenBSD}, \emph{NetBSD}, \emph{FreeBSD},
\emph{OpenSolaris}, \emph{OpenIndiana}, and
\emph{Debian} and \emph{Ubuntu} \emph{GNU/Linux}.
}
and Microsoft Windows.\footnote{
The \emph{Vista} and \emph{7} releases have been tested.
The binaries are built with the \emph{MinGW} tools, and
the the cost-free \emph{DMC} tools have been used too,
and as of this writing, will still build the code against
\emph{wxWidgets} 2.8.
}
Other platforms that are supported by \emph{wxWidgets} have
not been tested and
more work would certainly be needed to
build \IXpkg{} on those platforms
(excepting other Unix-like systems, which might
need only small changes if \emph{wxWidgets} is supported).
\IXpkgu{} has been
tested with \emph{wxWidgets} versions 2.6, 2.8, and the
development version 2.9.

%WX%2|Introduction|1000000||

% 2nd chapter, on the program
%
%WX%2|<R>Epspline.*: Prism and Lathe Editor|301000||
\chapter{\dtypkgu{}: Prism and Lathe Editor}

\section{Usage}%WX%3||303000||

\IXpkgu{} is an interactive program with a
graphical interface. Interaction is primarily by use of a mouse
(or any pointing device from which the windowing system delivers
`mouse' events to software).


\subsection{Overview}%WX%4||403000||
\subsection{Mouse and Key}%WX%4||404000||
% no coded comment for subsubsection
\subsubsection{Operations}%xxxWX%5||501000||
\subsection{Usage Table}%WX%4||405000||


\chapter{Licenses}%WX%2||202000||
% GNU doc license
\section{GNU General Public License}
	\label{sec:gpl_v3}
\begin{center}

Version 3, 29 June 2007

{\parindent 0in

Copyright \copyright\  2007 Free Software Foundation, Inc. \url{http://fsf.org/}

\bigskip
Everyone is permitted to copy and distribute verbatim copies of this

license document, but changing it is not allowed.}

\end{center}

%\begin{center}
\subsection*{\sc Preamble}%WX%4|PREAMBLE|||
%\end{center}
The GNU General Public License is a free, copyleft license for
software and other kinds of works.

The licenses for most software and other practical works are designed
to take away your freedom to share and change the works.  By contrast,
the GNU General Public License is intended to guarantee your freedom to
share and change all versions of a program--to make sure it remains free
software for all its users.  We, the Free Software Foundation, use the
GNU General Public License for most of our software; it applies also to
any other work released this way by its authors.  You can apply it to
your programs, too.

When we speak of free software, we are referring to freedom, not
price.  Our General Public Licenses are designed to make sure that you
have the freedom to distribute copies of free software (and charge for
them if you wish), that you receive source code or can get it if you
want it, that you can change the software or use pieces of it in new
free programs, and that you know you can do these things.

To protect your rights, we need to prevent others from denying you
these rights or asking you to surrender the rights.  Therefore, you have
certain responsibilities if you distribute copies of the software, or if
you modify it: responsibilities to respect the freedom of others.

For example, if you distribute copies of such a program, whether
gratis or for a fee, you must pass on to the recipients the same
freedoms that you received.  You must make sure that they, too, receive
or can get the source code.  And you must show them these terms so they
know their rights.

Developers that use the GNU GPL protect your rights with two steps:
(1) assert copyright on the software, and (2) offer you this License
giving you legal permission to copy, distribute and/or modify it.

For the developers' and authors' protection, the GPL clearly explains
that there is no warranty for this free software.  For both users' and
authors' sake, the GPL requires that modified versions be marked as
changed, so that their problems will not be attributed erroneously to
authors of previous versions.

Some devices are designed to deny users access to install or run
modified versions of the software inside them, although the manufacturer
can do so.  This is fundamentally incompatible with the aim of
protecting users' freedom to change the software.  The systematic
pattern of such abuse occurs in the area of products for individuals to
use, which is precisely where it is most unacceptable.  Therefore, we
have designed this version of the GPL to prohibit the practice for those
products.  If such problems arise substantially in other domains, we
stand ready to extend this provision to those domains in future versions
of the GPL, as needed to protect the freedom of users.

Finally, every program is threatened constantly by software patents.
States should not allow patents to restrict development and use of
software on general-purpose computers, but in those that do, we wish to
avoid the special danger that patents applied to a free program could
make it effectively proprietary.  To prevent this, the GPL assures that
patents cannot be used to render the program non-free.

The precise terms and conditions for copying, distribution and
modification follow.


%\begin{center}
\subsection*{\sc Terms and Conditions}%WX%4|TERMS AND CONDITIONS|||
%\end{center}

\begin{enumerate}

\addtocounter{enumi}{-1}

\item Definitions.

``This License'' refers to version 3 of the GNU General Public License.

``Copyright'' also means copyright-like laws that apply to other kinds of
works, such as semiconductor masks.

``The Program'' refers to any copyrightable work licensed under this
License.  Each licensee is addressed as ``you''.  ``Licensees'' and
``recipients'' may be individuals or organizations.

To ``modify'' a work means to copy from or adapt all or part of the work
in a fashion requiring copyright permission, other than the making of an
exact copy.  The resulting work is called a ``modified version'' of the
earlier work or a work ``based on'' the earlier work.

A ``covered work'' means either the unmodified Program or a work based
on the Program.

To ``propagate'' a work means to do anything with it that, without
permission, would make you directly or secondarily liable for
infringement under applicable copyright law, except executing it on a
computer or modifying a private copy.  Propagation includes copying,
distribution (with or without modification), making available to the
public, and in some countries other activities as well.

To ``convey'' a work means any kind of propagation that enables other
parties to make or receive copies.  Mere interaction with a user through
a computer network, with no transfer of a copy, is not conveying.

An interactive user interface displays ``Appropriate Legal Notices''
to the extent that it includes a convenient and prominently visible
feature that (1) displays an appropriate copyright notice, and (2)
tells the user that there is no warranty for the work (except to the
extent that warranties are provided), that licensees may convey the
work under this License, and how to view a copy of this License.  If
the interface presents a list of user commands or options, such as a
menu, a prominent item in the list meets this criterion.

\item Source Code.

The ``source code'' for a work means the preferred form of the work
for making modifications to it.  ``Object code'' means any non-source
form of a work.

A ``Standard Interface'' means an interface that either is an official
standard defined by a recognized standards body, or, in the case of
interfaces specified for a particular programming language, one that
is widely used among developers working in that language.

The ``System Libraries'' of an executable work include anything, other
than the work as a whole, that (a) is included in the normal form of
packaging a Major Component, but which is not part of that Major
Component, and (b) serves only to enable use of the work with that
Major Component, or to implement a Standard Interface for which an
implementation is available to the public in source code form.  A
``Major Component'', in this context, means a major essential component
(kernel, window system, and so on) of the specific operating system
(if any) on which the executable work runs, or a compiler used to
produce the work, or an object code interpreter used to run it.

The ``Corresponding Source'' for a work in object code form means all
the source code needed to generate, install, and (for an executable
work) run the object code and to modify the work, including scripts to
control those activities.  However, it does not include the work's
System Libraries, or general-purpose tools or generally available free
programs which are used unmodified in performing those activities but
which are not part of the work.  For example, Corresponding Source
includes interface definition files associated with source files for
the work, and the source code for shared libraries and dynamically
linked subprograms that the work is specifically designed to require,
such as by intimate data communication or control flow between those
subprograms and other parts of the work.

The Corresponding Source need not include anything that users
can regenerate automatically from other parts of the Corresponding
Source.

The Corresponding Source for a work in source code form is that
same work.

\item Basic Permissions.

All rights granted under this License are granted for the term of
copyright on the Program, and are irrevocable provided the stated
conditions are met.  This License explicitly affirms your unlimited
permission to run the unmodified Program.  The output from running a
covered work is covered by this License only if the output, given its
content, constitutes a covered work.  This License acknowledges your
rights of fair use or other equivalent, as provided by copyright law.

You may make, run and propagate covered works that you do not
convey, without conditions so long as your license otherwise remains
in force.  You may convey covered works to others for the sole purpose
of having them make modifications exclusively for you, or provide you
with facilities for running those works, provided that you comply with
the terms of this License in conveying all material for which you do
not control copyright.  Those thus making or running the covered works
for you must do so exclusively on your behalf, under your direction
and control, on terms that prohibit them from making any copies of
your copyrighted material outside their relationship with you.

Conveying under any other circumstances is permitted solely under
the conditions stated below.  Sublicensing is not allowed; section 10
makes it unnecessary.

\item Protecting Users' Legal Rights From Anti-Circumvention Law.

No covered work shall be deemed part of an effective technological
measure under any applicable law fulfilling obligations under article
11 of the WIPO copyright treaty adopted on 20 December 1996, or
similar laws prohibiting or restricting circumvention of such
measures.

When you convey a covered work, you waive any legal power to forbid
circumvention of technological measures to the extent such circumvention
is effected by exercising rights under this License with respect to
the covered work, and you disclaim any intention to limit operation or
modification of the work as a means of enforcing, against the work's
users, your or third parties' legal rights to forbid circumvention of
technological measures.

\item Conveying Verbatim Copies.

You may convey verbatim copies of the Program's source code as you
receive it, in any medium, provided that you conspicuously and
appropriately publish on each copy an appropriate copyright notice;
keep intact all notices stating that this License and any
non-permissive terms added in accord with section 7 apply to the code;
keep intact all notices of the absence of any warranty; and give all
recipients a copy of this License along with the Program.

You may charge any price or no price for each copy that you convey,
and you may offer support or warranty protection for a fee.

\item Conveying Modified Source Versions.

You may convey a work based on the Program, or the modifications to
produce it from the Program, in the form of source code under the
terms of section 4, provided that you also meet all of these conditions:
  \begin{enumerate}
  \item The work must carry prominent notices stating that you modified
  it, and giving a relevant date.

  \item The work must carry prominent notices stating that it is
  released under this License and any conditions added under section
  7.  This requirement modifies the requirement in section 4 to
  ``keep intact all notices''.

  \item You must license the entire work, as a whole, under this
  License to anyone who comes into possession of a copy.  This
  License will therefore apply, along with any applicable section 7
  additional terms, to the whole of the work, and all its parts,
  regardless of how they are packaged.  This License gives no
  permission to license the work in any other way, but it does not
  invalidate such permission if you have separately received it.

  \item If the work has interactive user interfaces, each must display
  Appropriate Legal Notices; however, if the Program has interactive
  interfaces that do not display Appropriate Legal Notices, your
  work need not make them do so.
\end{enumerate}
A compilation of a covered work with other separate and independent
works, which are not by their nature extensions of the covered work,
and which are not combined with it such as to form a larger program,
in or on a volume of a storage or distribution medium, is called an
``aggregate'' if the compilation and its resulting copyright are not
used to limit the access or legal rights of the compilation's users
beyond what the individual works permit.  Inclusion of a covered work
in an aggregate does not cause this License to apply to the other
parts of the aggregate.

\item Conveying Non-Source Forms.

You may convey a covered work in object code form under the terms
of sections 4 and 5, provided that you also convey the
machine-readable Corresponding Source under the terms of this License,
in one of these ways:
  \begin{enumerate}
  \item Convey the object code in, or embodied in, a physical product
  (including a physical distribution medium), accompanied by the
  Corresponding Source fixed on a durable physical medium
  customarily used for software interchange.

  \item Convey the object code in, or embodied in, a physical product
  (including a physical distribution medium), accompanied by a
  written offer, valid for at least three years and valid for as
  long as you offer spare parts or customer support for that product
  model, to give anyone who possesses the object code either (1) a
  copy of the Corresponding Source for all the software in the
  product that is covered by this License, on a durable physical
  medium customarily used for software interchange, for a price no
  more than your reasonable cost of physically performing this
  conveying of source, or (2) access to copy the
  Corresponding Source from a network server at no charge.

  \item Convey individual copies of the object code with a copy of the
  written offer to provide the Corresponding Source.  This
  alternative is allowed only occasionally and noncommercially, and
  only if you received the object code with such an offer, in accord
  with subsection 6b.

  \item Convey the object code by offering access from a designated
  place (gratis or for a charge), and offer equivalent access to the
  Corresponding Source in the same way through the same place at no
  further charge.  You need not require recipients to copy the
  Corresponding Source along with the object code.  If the place to
  copy the object code is a network server, the Corresponding Source
  may be on a different server (operated by you or a third party)
  that supports equivalent copying facilities, provided you maintain
  clear directions next to the object code saying where to find the
  Corresponding Source.  Regardless of what server hosts the
  Corresponding Source, you remain obligated to ensure that it is
  available for as long as needed to satisfy these requirements.

  \item Convey the object code using peer-to-peer transmission, provided
  you inform other peers where the object code and Corresponding
  Source of the work are being offered to the general public at no
  charge under subsection 6d.
  \end{enumerate}

A separable portion of the object code, whose source code is excluded
from the Corresponding Source as a System Library, need not be
included in conveying the object code work.

A ``User Product'' is either (1) a ``consumer product'', which means any
tangible personal property which is normally used for personal, family,
or household purposes, or (2) anything designed or sold for incorporation
into a dwelling.  In determining whether a product is a consumer product,
doubtful cases shall be resolved in favor of coverage.  For a particular
product received by a particular user, ``normally used'' refers to a
typical or common use of that class of product, regardless of the status
of the particular user or of the way in which the particular user
actually uses, or expects or is expected to use, the product.  A product
is a consumer product regardless of whether the product has substantial
commercial, industrial or non-consumer uses, unless such uses represent
the only significant mode of use of the product.

``Installation Information'' for a User Product means any methods,
procedures, authorization keys, or other information required to install
and execute modified versions of a covered work in that User Product from
a modified version of its Corresponding Source.  The information must
suffice to ensure that the continued functioning of the modified object
code is in no case prevented or interfered with solely because
modification has been made.

If you convey an object code work under this section in, or with, or
specifically for use in, a User Product, and the conveying occurs as
part of a transaction in which the right of possession and use of the
User Product is transferred to the recipient in perpetuity or for a
fixed term (regardless of how the transaction is characterized), the
Corresponding Source conveyed under this section must be accompanied
by the Installation Information.  But this requirement does not apply
if neither you nor any third party retains the ability to install
modified object code on the User Product (for example, the work has
been installed in ROM).

The requirement to provide Installation Information does not include a
requirement to continue to provide support service, warranty, or updates
for a work that has been modified or installed by the recipient, or for
the User Product in which it has been modified or installed.  Access to a
network may be denied when the modification itself materially and
adversely affects the operation of the network or violates the rules and
protocols for communication across the network.

Corresponding Source conveyed, and Installation Information provided,
in accord with this section must be in a format that is publicly
documented (and with an implementation available to the public in
source code form), and must require no special password or key for
unpacking, reading or copying.

\item Additional Terms.

``Additional permissions'' are terms that supplement the terms of this
License by making exceptions from one or more of its conditions.
Additional permissions that are applicable to the entire Program shall
be treated as though they were included in this License, to the extent
that they are valid under applicable law.  If additional permissions
apply only to part of the Program, that part may be used separately
under those permissions, but the entire Program remains governed by
this License without regard to the additional permissions.

When you convey a copy of a covered work, you may at your option
remove any additional permissions from that copy, or from any part of
it.  (Additional permissions may be written to require their own
removal in certain cases when you modify the work.)  You may place
additional permissions on material, added by you to a covered work,
for which you have or can give appropriate copyright permission.

Notwithstanding any other provision of this License, for material you
add to a covered work, you may (if authorized by the copyright holders of
that material) supplement the terms of this License with terms:
  \begin{enumerate}
  \item Disclaiming warranty or limiting liability differently from the
  terms of sections 15 and 16 of this License; or

  \item Requiring preservation of specified reasonable legal notices or
  author attributions in that material or in the Appropriate Legal
  Notices displayed by works containing it; or

  \item Prohibiting misrepresentation of the origin of that material, or
  requiring that modified versions of such material be marked in
  reasonable ways as different from the original version; or

  \item Limiting the use for publicity purposes of names of licensors or
  authors of the material; or

  \item Declining to grant rights under trademark law for use of some
  trade names, trademarks, or service marks; or

  \item Requiring indemnification of licensors and authors of that
  material by anyone who conveys the material (or modified versions of
  it) with contractual assumptions of liability to the recipient, for
  any liability that these contractual assumptions directly impose on
  those licensors and authors.
  \end{enumerate}

All other non-permissive additional terms are considered ``further
restrictions'' within the meaning of section 10.  If the Program as you
received it, or any part of it, contains a notice stating that it is
governed by this License along with a term that is a further
restriction, you may remove that term.  If a license document contains
a further restriction but permits relicensing or conveying under this
License, you may add to a covered work material governed by the terms
of that license document, provided that the further restriction does
not survive such relicensing or conveying.

If you add terms to a covered work in accord with this section, you
must place, in the relevant source files, a statement of the
additional terms that apply to those files, or a notice indicating
where to find the applicable terms.

Additional terms, permissive or non-permissive, may be stated in the
form of a separately written license, or stated as exceptions;
the above requirements apply either way.

\item Termination.

You may not propagate or modify a covered work except as expressly
provided under this License.  Any attempt otherwise to propagate or
modify it is void, and will automatically terminate your rights under
this License (including any patent licenses granted under the third
paragraph of section 11).

However, if you cease all violation of this License, then your
license from a particular copyright holder is reinstated (a)
provisionally, unless and until the copyright holder explicitly and
finally terminates your license, and (b) permanently, if the copyright
holder fails to notify you of the violation by some reasonable means
prior to 60 days after the cessation.

Moreover, your license from a particular copyright holder is
reinstated permanently if the copyright holder notifies you of the
violation by some reasonable means, this is the first time you have
received notice of violation of this License (for any work) from that
copyright holder, and you cure the violation prior to 30 days after
your receipt of the notice.

Termination of your rights under this section does not terminate the
licenses of parties who have received copies or rights from you under
this License.  If your rights have been terminated and not permanently
reinstated, you do not qualify to receive new licenses for the same
material under section 10.

\item Acceptance Not Required for Having Copies.

You are not required to accept this License in order to receive or
run a copy of the Program.  Ancillary propagation of a covered work
occurring solely as a consequence of using peer-to-peer transmission
to receive a copy likewise does not require acceptance.  However,
nothing other than this License grants you permission to propagate or
modify any covered work.  These actions infringe copyright if you do
not accept this License.  Therefore, by modifying or propagating a
covered work, you indicate your acceptance of this License to do so.

\item Automatic Licensing of Downstream Recipients.

Each time you convey a covered work, the recipient automatically
receives a license from the original licensors, to run, modify and
propagate that work, subject to this License.  You are not responsible
for enforcing compliance by third parties with this License.

An ``entity transaction'' is a transaction transferring control of an
organization, or substantially all assets of one, or subdividing an
organization, or merging organizations.  If propagation of a covered
work results from an entity transaction, each party to that
transaction who receives a copy of the work also receives whatever
licenses to the work the party's predecessor in interest had or could
give under the previous paragraph, plus a right to possession of the
Corresponding Source of the work from the predecessor in interest, if
the predecessor has it or can get it with reasonable efforts.

You may not impose any further restrictions on the exercise of the
rights granted or affirmed under this License.  For example, you may
not impose a license fee, royalty, or other charge for exercise of
rights granted under this License, and you may not initiate litigation
(including a cross-claim or counterclaim in a lawsuit) alleging that
any patent claim is infringed by making, using, selling, offering for
sale, or importing the Program or any portion of it.

\item Patents.

A ``contributor'' is a copyright holder who authorizes use under this
License of the Program or a work on which the Program is based.  The
work thus licensed is called the contributor's ``contributor version''.

A contributor's ``essential patent claims'' are all patent claims
owned or controlled by the contributor, whether already acquired or
hereafter acquired, that would be infringed by some manner, permitted
by this License, of making, using, or selling its contributor version,
but do not include claims that would be infringed only as a
consequence of further modification of the contributor version.  For
purposes of this definition, ``control'' includes the right to grant
patent sublicenses in a manner consistent with the requirements of
this License.

Each contributor grants you a non-exclusive, worldwide, royalty-free
patent license under the contributor's essential patent claims, to
make, use, sell, offer for sale, import and otherwise run, modify and
propagate the contents of its contributor version.

In the following three paragraphs, a ``patent license'' is any express
agreement or commitment, however denominated, not to enforce a patent
(such as an express permission to practice a patent or covenant not to
sue for patent infringement).  To ``grant'' such a patent license to a
party means to make such an agreement or commitment not to enforce a
patent against the party.

If you convey a covered work, knowingly relying on a patent license,
and the Corresponding Source of the work is not available for anyone
to copy, free of charge and under the terms of this License, through a
publicly available network server or other readily accessible means,
then you must either (1) cause the Corresponding Source to be so
available, or (2) arrange to deprive yourself of the benefit of the
patent license for this particular work, or (3) arrange, in a manner
consistent with the requirements of this License, to extend the patent
license to downstream recipients.  ``Knowingly relying'' means you have
actual knowledge that, but for the patent license, your conveying the
covered work in a country, or your recipient's use of the covered work
in a country, would infringe one or more identifiable patents in that
country that you have reason to believe are valid.

If, pursuant to or in connection with a single transaction or
arrangement, you convey, or propagate by procuring conveyance of, a
covered work, and grant a patent license to some of the parties
receiving the covered work authorizing them to use, propagate, modify
or convey a specific copy of the covered work, then the patent license
you grant is automatically extended to all recipients of the covered
work and works based on it.

A patent license is ``discriminatory'' if it does not include within
the scope of its coverage, prohibits the exercise of, or is
conditioned on the non-exercise of one or more of the rights that are
specifically granted under this License.  You may not convey a covered
work if you are a party to an arrangement with a third party that is
in the business of distributing software, under which you make payment
to the third party based on the extent of your activity of conveying
the work, and under which the third party grants, to any of the
parties who would receive the covered work from you, a discriminatory
patent license (a) in connection with copies of the covered work
conveyed by you (or copies made from those copies), or (b) primarily
for and in connection with specific products or compilations that
contain the covered work, unless you entered into that arrangement,
or that patent license was granted, prior to 28 March 2007.

Nothing in this License shall be construed as excluding or limiting
any implied license or other defenses to infringement that may
otherwise be available to you under applicable patent law.

\item No Surrender of Others' Freedom.

If conditions are imposed on you (whether by court order, agreement or
otherwise) that contradict the conditions of this License, they do not
excuse you from the conditions of this License.  If you cannot convey a
covered work so as to satisfy simultaneously your obligations under this
License and any other pertinent obligations, then as a consequence you may
not convey it at all.  For example, if you agree to terms that obligate you
to collect a royalty for further conveying from those to whom you convey
the Program, the only way you could satisfy both those terms and this
License would be to refrain entirely from conveying the Program.

\item Use with the GNU Affero General Public License.

Notwithstanding any other provision of this License, you have
permission to link or combine any covered work with a work licensed
under version 3 of the GNU Affero General Public License into a single
combined work, and to convey the resulting work.  The terms of this
License will continue to apply to the part which is the covered work,
but the special requirements of the GNU Affero General Public License,
section 13, concerning interaction through a network will apply to the
combination as such.

\item Revised Versions of this License.

The Free Software Foundation may publish revised and/or new versions of
the GNU General Public License from time to time.  Such new versions will
be similar in spirit to the present version, but may differ in detail to
address new problems or concerns.

Each version is given a distinguishing version number.  If the
Program specifies that a certain numbered version of the GNU General
Public License ``or any later version'' applies to it, you have the
option of following the terms and conditions either of that numbered
version or of any later version published by the Free Software
Foundation.  If the Program does not specify a version number of the
GNU General Public License, you may choose any version ever published
by the Free Software Foundation.

If the Program specifies that a proxy can decide which future
versions of the GNU General Public License can be used, that proxy's
public statement of acceptance of a version permanently authorizes you
to choose that version for the Program.

Later license versions may give you additional or different
permissions.  However, no additional obligations are imposed on any
author or copyright holder as a result of your choosing to follow a
later version.

\item Disclaimer of Warranty.

\begin{sloppypar}
 THERE IS NO WARRANTY FOR THE PROGRAM, TO THE EXTENT PERMITTED BY
 APPLICABLE LAW.  EXCEPT WHEN OTHERWISE STATED IN WRITING THE
 COPYRIGHT HOLDERS AND/OR OTHER PARTIES PROVIDE THE PROGRAM ``AS IS''
 WITHOUT WARRANTY OF ANY KIND, EITHER EXPRESSED OR IMPLIED,
 INCLUDING, BUT NOT LIMITED TO, THE IMPLIED WARRANTIES OF
 MERCHANTABILITY AND FITNESS FOR A PARTICULAR PURPOSE.  THE ENTIRE
 RISK AS TO THE QUALITY AND PERFORMANCE OF THE PROGRAM IS WITH YOU.
 SHOULD THE PROGRAM PROVE DEFECTIVE, YOU ASSUME THE COST OF ALL
 NECESSARY SERVICING, REPAIR OR CORRECTION.
\end{sloppypar}

\item Limitation of Liability.

 IN NO EVENT UNLESS REQUIRED BY APPLICABLE LAW OR AGREED TO IN
 WRITING WILL ANY COPYRIGHT HOLDER, OR ANY OTHER PARTY WHO MODIFIES
 AND/OR CONVEYS THE PROGRAM AS PERMITTED ABOVE, BE LIABLE TO YOU FOR
 DAMAGES, INCLUDING ANY GENERAL, SPECIAL, INCIDENTAL OR CONSEQUENTIAL
 DAMAGES ARISING OUT OF THE USE OR INABILITY TO USE THE PROGRAM
 (INCLUDING BUT NOT LIMITED TO LOSS OF DATA OR DATA BEING RENDERED
 INACCURATE OR LOSSES SUSTAINED BY YOU OR THIRD PARTIES OR A FAILURE
 OF THE PROGRAM TO OPERATE WITH ANY OTHER PROGRAMS), EVEN IF SUCH
 HOLDER OR OTHER PARTY HAS BEEN ADVISED OF THE POSSIBILITY OF SUCH
 DAMAGES.

\item Interpretation of Sections 15 and 16.

If the disclaimer of warranty and limitation of liability provided
above cannot be given local legal effect according to their terms,
reviewing courts shall apply local law that most closely approximates
an absolute waiver of all civil liability in connection with the
Program, unless a warranty or assumption of liability accompanies a
copy of the Program in return for a fee.

%\end{enumerate}

%\begin{center}
{\Large\sc End of Terms and Conditions}
%\end{center}

\bigskip
%\begin{center}
\subsection*{\sc How to Apply These Terms to Your New Programs}%WX%4|APPLY THESE TERMS TO YOUR NEW PROGRAMS|||
%\end{center}

If you develop a new program, and you want it to be of the greatest
possible use to the public, the best way to achieve this is to make it
free software which everyone can redistribute and change under these terms.

To do so, attach the following notices to the program.  It is safest
to attach them to the start of each source file to most effectively
state the exclusion of warranty; and each file should have at least
the ``copyright'' line and a pointer to where the full notice is found.

{\footnotesize
\begin{verbatim}
<one line to give the program's name and a brief idea of what it does.>

Copyright (C) <textyear>  <name of author>

This program is free software: you can redistribute it and/or modify
it under the terms of the GNU General Public License as published by
the Free Software Foundation, either version 3 of the License, or
(at your option) any later version.

This program is distributed in the hope that it will be useful,
but WITHOUT ANY WARRANTY; without even the implied warranty of
MERCHANTABILITY or FITNESS FOR A PARTICULAR PURPOSE.  See the
GNU General Public License for more details.

You should have received a copy of the GNU General Public License
along with this program.  If not, see <http://www.gnu.org/licenses/>.
\end{verbatim}
}

Also add information on how to contact you by electronic and paper mail.

If the program does terminal interaction, make it output a short
notice like this when it starts in an interactive mode:

{\footnotesize
\begin{verbatim}
<program>  Copyright (C) <year>  <name of author>

This program comes with ABSOLUTELY NO WARRANTY; for details type `show w'.
This is free software, and you are welcome to redistribute it
under certain conditions; type `show c' for details.
\end{verbatim}
}

The hypothetical commands {\tt show w} and {\tt show c} should show
the appropriate
parts of the General Public License.  Of course, your program's commands
might be different; for a GUI interface, you would use an ``about box''.

You should also get your employer (if you work as a programmer) or
school, if any, to sign a ``copyright disclaimer'' for the program, if
necessary.  For more information on this, and how to apply and follow
the GNU GPL, see \url{http://www.gnu.org/licenses/}.

The GNU General Public License does not permit incorporating your
program into proprietary programs.  If your program is a subroutine
library, you may consider it more useful to permit linking proprietary
applications with the library.  If this is what you want to do, use
the GNU Lesser General Public License instead of this License.  But
first, please read \url{http://www.gnu.org/philosophy/why-not-lgpl.html}.

\end{enumerate}

%WX%3|GNU General Public License||GPL3_Section|
\section{GNU General Public License Version 2}
	\label{sec:gpl_v2}
\begin{center}

Version 2, June 1991

{\parindent 0in

Copyright \copyright\ 1989, 1991 Free Software Foundation, Inc.

\bigskip

51 Franklin Street, Fifth Floor, Boston, MA  02110-1301, USA

\bigskip

Everyone is permitted to copy and distribute verbatim copies
of this license document, but changing it is not allowed.
}
\end{center}

%\begin{center}
\subsection*{Preamble}%WX%4|||||
%\end{center}


The licenses for most software are designed to take away your freedom to
share and change it.  By contrast, the GNU General Public License is
intended to guarantee your freedom to share and change free software---to
make sure the software is free for all its users.  This General Public
License applies to most of the Free Software Foundation's software and to
any other program whose authors commit to using it.  (Some other Free
Software Foundation software is covered by the GNU Library General Public
License instead.)  You can apply it to your programs, too.

When we speak of free software, we are referring to freedom, not price.
Our General Public Licenses are designed to make sure that you have the
freedom to distribute copies of free software (and charge for this service
if you wish), that you receive source code or can get it if you want it,
that you can change the software or use pieces of it in new free programs;
and that you know you can do these things.

To protect your rights, we need to make restrictions that forbid anyone to
deny you these rights or to ask you to surrender the rights.  These
restrictions translate to certain responsibilities for you if you
distribute copies of the software, or if you modify it.

For example, if you distribute copies of such a program, whether gratis or
for a fee, you must give the recipients all the rights that you have.  You
must make sure that they, too, receive or can get the source code.  And
you must show them these terms so they know their rights.

We protect your rights with two steps: (1) copyright the software, and (2)
offer you this license which gives you legal permission to copy,
distribute and/or modify the software.

Also, for each author's protection and ours, we want to make certain that
everyone understands that there is no warranty for this free software.  If
the software is modified by someone else and passed on, we want its
recipients to know that what they have is not the original, so that any
problems introduced by others will not reflect on the original authors'
reputations.

Finally, any free program is threatened constantly by software patents.
We wish to avoid the danger that redistributors of a free program will
individually obtain patent licenses, in effect making the program
proprietary.  To prevent this, we have made it clear that any patent must
be licensed for everyone's free use or not licensed at all.

The precise terms and conditions for copying, distribution and
modification follow.

%\begin{center}
\subsection*{Terms and Conditions For Copying, Distribution and Modification}%WX%4||||
%\end{center}


%\renewcommand{\theenumi}{\alpha{enumi}}
\begin{enumerate}

\addtocounter{enumi}{-1}

\item 

This License applies to any program or other work which contains a notice
placed by the copyright holder saying it may be distributed under the
terms of this General Public License.  The ``Program'', below, refers to
any such program or work, and a ``work based on the Program'' means either
the Program or any derivative work under copyright law: that is to say, a
work containing the Program or a portion of it, either verbatim or with
modifications and/or translated into another language.  (Hereinafter,
translation is included without limitation in the term ``modification''.)
Each licensee is addressed as ``you''.

Activities other than copying, distribution and modification are not
covered by this License; they are outside its scope.  The act of
running the Program is not restricted, and the output from the Program
is covered only if its contents constitute a work based on the
Program (independent of having been made by running the Program).
Whether that is true depends on what the Program does.

\item You may copy and distribute verbatim copies of the Program's source
  code as you receive it, in any medium, provided that you conspicuously
  and appropriately publish on each copy an appropriate copyright notice
  and disclaimer of warranty; keep intact all the notices that refer to
  this License and to the absence of any warranty; and give any other
  recipients of the Program a copy of this License along with the Program.

You may charge a fee for the physical act of transferring a copy, and you
may at your option offer warranty protection in exchange for a fee.

\item

You may modify your copy or copies of the Program or any portion
of it, thus forming a work based on the Program, and copy and
distribute such modifications or work under the terms of Section 1
above, provided that you also meet all of these conditions:

\begin{enumerate}

\item 

You must cause the modified files to carry prominent notices stating that
you changed the files and the date of any change.

\item

You must cause any work that you distribute or publish, that in
whole or in part contains or is derived from the Program or any
part thereof, to be licensed as a whole at no charge to all third
parties under the terms of this License.

\item
If the modified program normally reads commands interactively
when run, you must cause it, when started running for such
interactive use in the most ordinary way, to print or display an
announcement including an appropriate copyright notice and a
notice that there is no warranty (or else, saying that you provide
a warranty) and that users may redistribute the program under
these conditions, and telling the user how to view a copy of this
License.  (Exception: if the Program itself is interactive but
does not normally print such an announcement, your work based on
the Program is not required to print an announcement.)

\end{enumerate}


These requirements apply to the modified work as a whole.  If
identifiable sections of that work are not derived from the Program,
and can be reasonably considered independent and separate works in
themselves, then this License, and its terms, do not apply to those
sections when you distribute them as separate works.  But when you
distribute the same sections as part of a whole which is a work based
on the Program, the distribution of the whole must be on the terms of
this License, whose permissions for other licensees extend to the
entire whole, and thus to each and every part regardless of who wrote it.

Thus, it is not the intent of this section to claim rights or contest
your rights to work written entirely by you; rather, the intent is to
exercise the right to control the distribution of derivative or
collective works based on the Program.

In addition, mere aggregation of another work not based on the Program
with the Program (or with a work based on the Program) on a volume of
a storage or distribution medium does not bring the other work under
the scope of this License.

\item
You may copy and distribute the Program (or a work based on it,
under Section 2) in object code or executable form under the terms of
Sections 1 and 2 above provided that you also do one of the following:

\begin{enumerate}

\item

Accompany it with the complete corresponding machine-readable
source code, which must be distributed under the terms of Sections
1 and 2 above on a medium customarily used for software interchange; or,

\item

Accompany it with a written offer, valid for at least three
years, to give any third party, for a charge no more than your
cost of physically performing source distribution, a complete
machine-readable copy of the corresponding source code, to be
distributed under the terms of Sections 1 and 2 above on a medium
customarily used for software interchange; or,

\item

Accompany it with the information you received as to the offer
to distribute corresponding source code.  (This alternative is
allowed only for noncommercial distribution and only if you
received the program in object code or executable form with such
an offer, in accord with Subsection b above.)

\end{enumerate}


The source code for a work means the preferred form of the work for
making modifications to it.  For an executable work, complete source
code means all the source code for all modules it contains, plus any
associated interface definition files, plus the scripts used to
control compilation and installation of the executable.  However, as a
special exception, the source code distributed need not include
anything that is normally distributed (in either source or binary
form) with the major components (compiler, kernel, and so on) of the
operating system on which the executable runs, unless that component
itself accompanies the executable.

If distribution of executable or object code is made by offering
access to copy from a designated place, then offering equivalent
access to copy the source code from the same place counts as
distribution of the source code, even though third parties are not
compelled to copy the source along with the object code.

\item
You may not copy, modify, sublicense, or distribute the Program
except as expressly provided under this License.  Any attempt
otherwise to copy, modify, sublicense or distribute the Program is
void, and will automatically terminate your rights under this License.
However, parties who have received copies, or rights, from you under
this License will not have their licenses terminated so long as such
parties remain in full compliance.

\item
You are not required to accept this License, since you have not
signed it.  However, nothing else grants you permission to modify or
distribute the Program or its derivative works.  These actions are
prohibited by law if you do not accept this License.  Therefore, by
modifying or distributing the Program (or any work based on the
Program), you indicate your acceptance of this License to do so, and
all its terms and conditions for copying, distributing or modifying
the Program or works based on it.

\item
Each time you redistribute the Program (or any work based on the
Program), the recipient automatically receives a license from the
original licensor to copy, distribute or modify the Program subject to
these terms and conditions.  You may not impose any further
restrictions on the recipients' exercise of the rights granted herein.
You are not responsible for enforcing compliance by third parties to
this License.

\item
If, as a consequence of a court judgment or allegation of patent
infringement or for any other reason (not limited to patent issues),
conditions are imposed on you (whether by court order, agreement or
otherwise) that contradict the conditions of this License, they do not
excuse you from the conditions of this License.  If you cannot
distribute so as to satisfy simultaneously your obligations under this
License and any other pertinent obligations, then as a consequence you
may not distribute the Program at all.  For example, if a patent
license would not permit royalty-free redistribution of the Program by
all those who receive copies directly or indirectly through you, then
the only way you could satisfy both it and this License would be to
refrain entirely from distribution of the Program.

If any portion of this section is held invalid or unenforceable under
any particular circumstance, the balance of the section is intended to
apply and the section as a whole is intended to apply in other
circumstances.

It is not the purpose of this section to induce you to infringe any
patents or other property right claims or to contest validity of any
such claims; this section has the sole purpose of protecting the
integrity of the free software distribution system, which is
implemented by public license practices.  Many people have made
generous contributions to the wide range of software distributed
through that system in reliance on consistent application of that
system; it is up to the author/donor to decide if he or she is willing
to distribute software through any other system and a licensee cannot
impose that choice.

This section is intended to make thoroughly clear what is believed to
be a consequence of the rest of this License.

\item
If the distribution and/or use of the Program is restricted in
certain countries either by patents or by copyrighted interfaces, the
original copyright holder who places the Program under this License
may add an explicit geographical distribution limitation excluding
those countries, so that distribution is permitted only in or among
countries not thus excluded.  In such case, this License incorporates
the limitation as if written in the body of this License.

\item
The Free Software Foundation may publish revised and/or new versions
of the General Public License from time to time.  Such new versions will
be similar in spirit to the present version, but may differ in detail to
address new problems or concerns.

Each version is given a distinguishing version number.  If the Program
specifies a version number of this License which applies to it and ``any
later version'', you have the option of following the terms and conditions
either of that version or of any later version published by the Free
Software Foundation.  If the Program does not specify a version number of
this License, you may choose any version ever published by the Free Software
Foundation.

\item
If you wish to incorporate parts of the Program into other free
programs whose distribution conditions are different, write to the author
to ask for permission.  For software which is copyrighted by the Free
Software Foundation, write to the Free Software Foundation; we sometimes
make exceptions for this.  Our decision will be guided by the two goals
of preserving the free status of all derivatives of our free software and
of promoting the sharing and reuse of software generally.

%\begin{center}
{\bf No Warranty}
%\end{center}

\item
{\sc Because the program is licensed free of charge, there is no warranty
for the program, to the extent permitted by applicable law.  Except when
otherwise stated in writing the copyright holders and/or other parties
provide the program ``as is'' without warranty of any kind, either expressed
or implied, including, but not limited to, the implied warranties of
merchantability and fitness for a particular purpose.  The entire risk as
to the quality and performance of the program is with you.  Should the
program prove defective, you assume the cost of all necessary servicing,
repair or correction.}

\item
{\sc In no event unless required by applicable law or agreed to in writing
will any copyright holder, or any other party who may modify and/or
redistribute the program as permitted above, be liable to you for damages,
including any general, special, incidental or consequential damages arising
out of the use or inability to use the program (including but not limited
to loss of data or data being rendered inaccurate or losses sustained by
you or third parties or a failure of the program to operate with any other
programs), even if such holder or other party has been advised of the
possibility of such damages.}

\end{enumerate}


%\begin{center}
{\bf End of Terms and Conditions}
%\end{center}


%\pagebreak[2]

\subsection*{Appendix: How to Apply These Terms to Your New Programs}%WX%4|How to Apply These Terms to Your New Programs|||

If you develop a new program, and you want it to be of the greatest
possible use to the public, the best way to achieve this is to make it
free software which everyone can redistribute and change under these
terms.

  To do so, attach the following notices to the program.  It is safest to
  attach them to the start of each source file to most effectively convey
  the exclusion of warranty; and each file should have at least the
  ``copyright'' line and a pointer to where the full notice is found.

\begin{quote}
one line to give the program's name and a brief idea of what it does. \\
Copyright (C) yyyy  name of author \\

This program is free software; you can redistribute it and/or modify
it under the terms of the GNU General Public License as published by
the Free Software Foundation; either version 2 of the License, or
(at your option) any later version.

This program is distributed in the hope that it will be useful,
but WITHOUT ANY WARRANTY; without even the implied warranty of
MERCHANTABILITY or FITNESS FOR A PARTICULAR PURPOSE.  See the
GNU General Public License for more details.

You should have received a copy of the GNU General Public License
along with this program; if not, write to the Free Software
Foundation, Inc., 51 Franklin Street, Fifth Floor, Boston, MA  02110-1301, USA.
\end{quote}

Also add information on how to contact you by electronic and paper mail.

If the program is interactive, make it output a short notice like this
when it starts in an interactive mode:

\begin{quote}
Gnomovision version 69, Copyright (C) yyyy  name of author \\
Gnomovision comes with ABSOLUTELY NO WARRANTY; for details type `show w'. \\
This is free software, and you are welcome to redistribute it
under certain conditions; type `show c' for details.
\end{quote}


The hypothetical commands {\tt show w} and {\tt show c} should show the
appropriate parts of the General Public License.  Of course, the commands
you use may be called something other than {\tt show w} and {\tt show c};
they could even be mouse-clicks or menu items---whatever suits your
program.

You should also get your employer (if you work as a programmer) or your
school, if any, to sign a ``copyright disclaimer'' for the program, if
necessary.  Here is a sample; alter the names:

\begin{quote}
Yoyodyne, Inc., hereby disclaims all copyright interest in the program \\
`Gnomovision' (which makes passes at compilers) written by James Hacker. \\

signature of Ty Coon, 1 April 1989 \\
Ty Coon, President of Vice
\end{quote}


This General Public License does not permit incorporating your program
into proprietary programs.  If your program is a subroutine library, you
may consider it more useful to permit linking proprietary applications
with the library.  If this is what you want to do, use the GNU Library
General Public License instead of this License.

%WX%3|GNU General Public License Version 2||GPL2_Section|
%---------------------------------------------------------------------
\section{GNU Free Documentation License}
%\label{label_fdl}

 \begin{center}

       Version 1.2, November 2002


 Copyright \copyright 2000,2001,2002  Free Software Foundation, Inc.
 
 \bigskip
 
     59 Temple Place, Suite 330, Boston, MA  02111-1307  USA
  
 \bigskip
 
 Everyone is permitted to copy and distribute verbatim copies
 of this license document, but changing it is not allowed.
\end{center}


%\begin{center}
\subsection*{Preamble}%WX%4|Preamble|||
%\end{center}

The purpose of this License is to make a manual, textbook, or other
functional and useful document ``free'' in the sense of freedom: to
assure everyone the effective freedom to copy and redistribute it,
with or without modifying it, either commercially or noncommercially.
Secondarily, this License preserves for the author and publisher a way
to get credit for their work, while not being considered responsible
for modifications made by others.

This License is a kind of ``copyleft'', which means that derivative
works of the document must themselves be free in the same sense.  It
complements the GNU General Public License, which is a copyleft
license designed for free software.

We have designed this License in order to use it for manuals for free
software, because free software needs free documentation: a free
program should come with manuals providing the same freedoms that the
software does.  But this License is not limited to software manuals;
it can be used for any textual work, regardless of subject matter or
whether it is published as a printed book.  We recommend this License
principally for works whose purpose is instruction or reference.


%\begin{center}
\subsection*{1. APPLICABILITY AND DEFINITIONS}%WX%4|1. APPLICABILITY AND DEFINITIONS|||
%\addcontentsline{toc}{section}{1. APPLICABILITY AND DEFINITIONS}
%\end{center}

This License applies to any manual or other work, in any medium, that
contains a notice placed by the copyright holder saying it can be
distributed under the terms of this License.  Such a notice grants a
world-wide, royalty-free license, unlimited in duration, to use that
work under the conditions stated herein.  The \textbf{``Document''}, below,
refers to any such manual or work.  Any member of the public is a
licensee, and is addressed as \textbf{``you''}.  You accept the license if you
copy, modify or distribute the work in a way requiring permission
under copyright law.

A \textbf{``Modified Version''} of the Document means any work containing the
Document or a portion of it, either copied verbatim, or with
modifications and/or translated into another language.

A \textbf{``Secondary Section''} is a named appendix or a front-matter section of
the Document that deals exclusively with the relationship of the
publishers or authors of the Document to the Document's overall subject
(or to related matters) and contains nothing that could fall directly
within that overall subject.  (Thus, if the Document is in part a
textbook of mathematics, a Secondary Section may not explain any
mathematics.)  The relationship could be a matter of historical
connection with the subject or with related matters, or of legal,
commercial, philosophical, ethical or political position regarding
them.

The \textbf{``Invariant Sections''} are certain Secondary Sections whose titles
are designated, as being those of Invariant Sections, in the notice
that says that the Document is released under this License.  If a
section does not fit the above definition of Secondary then it is not
allowed to be designated as Invariant.  The Document may contain zero
Invariant Sections.  If the Document does not identify any Invariant
Sections then there are none.

The \textbf{``Cover Texts''} are certain short passages of text that are listed,
as Front-Cover Texts or Back-Cover Texts, in the notice that says that
the Document is released under this License.  A Front-Cover Text may
be at most 5 words, and a Back-Cover Text may be at most 25 words.

A \textbf{``Transparent''} copy of the Document means a machine-readable copy,
represented in a format whose specification is available to the
general public, that is suitable for revising the document
straightforwardly with generic text editors or (for images composed of
pixels) generic paint programs or (for drawings) some widely available
drawing editor, and that is suitable for input to text formatters or
for automatic translation to a variety of formats suitable for input
to text formatters.  A copy made in an otherwise Transparent file
format whose markup, or absence of markup, has been arranged to thwart
or discourage subsequent modification by readers is not Transparent.
An image format is not Transparent if used for any substantial amount
of text.  A copy that is not ``Transparent'' is called \textbf{``Opaque''}.

Examples of suitable formats for Transparent copies include plain
ASCII without markup, Texinfo input format, LaTeX input format, SGML
or XML using a publicly available DTD, and standard-conforming simple
HTML, PostScript or PDF designed for human modification.  Examples of
transparent image formats include PNG, XCF and JPG.  Opaque formats
include proprietary formats that can be read and edited only by
proprietary word processors, SGML or XML for which the DTD and/or
processing tools are not generally available, and the
machine-generated HTML, PostScript or PDF produced by some word
processors for output purposes only.

The \textbf{``Title Page''} means, for a printed book, the title page itself,
plus such following pages as are needed to hold, legibly, the material
this License requires to appear in the title page.  For works in
formats which do not have any title page as such, ``Title Page'' means
the text near the most prominent appearance of the work's title,
preceding the beginning of the body of the text.

A section \textbf{``Entitled XYZ''} means a named subunit of the Document whose
title either is precisely XYZ or contains XYZ in parentheses following
text that translates XYZ in another language.  (Here XYZ stands for a
specific section name mentioned below, such as \textbf{``Acknowledgements''},
\textbf{``Dedications''}, \textbf{``Endorsements''}, or \textbf{``History''}.)  
To \textbf{``Preserve the Title''}
of such a section when you modify the Document means that it remains a
section ``Entitled XYZ'' according to this definition.

The Document may include Warranty Disclaimers next to the notice which
states that this License applies to the Document.  These Warranty
Disclaimers are considered to be included by reference in this
License, but only as regards disclaiming warranties: any other
implication that these Warranty Disclaimers may have is void and has
no effect on the meaning of this License.


%\begin{center}
\subsection*{2. VERBATIM COPYING}%WX%4|2. VERBATIM COPYING|||
%\addcontentsline{toc}{section}{2. VERBATIM COPYING}
%\end{center}

You may copy and distribute the Document in any medium, either
commercially or noncommercially, provided that this License, the
copyright notices, and the license notice saying this License applies
to the Document are reproduced in all copies, and that you add no other
conditions whatsoever to those of this License.  You may not use
technical measures to obstruct or control the reading or further
copying of the copies you make or distribute.  However, you may accept
compensation in exchange for copies.  If you distribute a large enough
number of copies you must also follow the conditions in section 3.

You may also lend copies, under the same conditions stated above, and
you may publicly display copies.


%\begin{center}
\subsection*{3. COPYING IN QUANTITY}%WX%4|3. COPYING IN QUANTITY|||
%\addcontentsline{toc}{section}{3. COPYING IN QUANTITY}
%\end{center}


If you publish printed copies (or copies in media that commonly have
printed covers) of the Document, numbering more than 100, and the
Document's license notice requires Cover Texts, you must enclose the
copies in covers that carry, clearly and legibly, all these Cover
Texts: Front-Cover Texts on the front cover, and Back-Cover Texts on
the back cover.  Both covers must also clearly and legibly identify
you as the publisher of these copies.  The front cover must present
the full title with all words of the title equally prominent and
visible.  You may add other material on the covers in addition.
Copying with changes limited to the covers, as long as they preserve
the title of the Document and satisfy these conditions, can be treated
as verbatim copying in other respects.

If the required texts for either cover are too voluminous to fit
legibly, you should put the first ones listed (as many as fit
reasonably) on the actual cover, and continue the rest onto adjacent
pages.

If you publish or distribute Opaque copies of the Document numbering
more than 100, you must either include a machine-readable Transparent
copy along with each Opaque copy, or state in or with each Opaque copy
a computer-network location from which the general network-using
public has access to download using public-standard network protocols
a complete Transparent copy of the Document, free of added material.
If you use the latter option, you must take reasonably prudent steps,
when you begin distribution of Opaque copies in quantity, to ensure
that this Transparent copy will remain thus accessible at the stated
location until at least one year after the last time you distribute an
Opaque copy (directly or through your agents or retailers) of that
edition to the public.

It is requested, but not required, that you contact the authors of the
Document well before redistributing any large number of copies, to give
them a chance to provide you with an updated version of the Document.


%\begin{center}
\subsection*{4. MODIFICATIONS}%WX%4|4. MODIFICATIONS|||
%\addcontentsline{toc}{section}{4. MODIFICATIONS}
%\end{center}

You may copy and distribute a Modified Version of the Document under
the conditions of sections 2 and 3 above, provided that you release
the Modified Version under precisely this License, with the Modified
Version filling the role of the Document, thus licensing distribution
and modification of the Modified Version to whoever possesses a copy
of it.  In addition, you must do these things in the Modified Version:

\begin{itemize}
\item[A.] 
   Use in the Title Page (and on the covers, if any) a title distinct
   from that of the Document, and from those of previous versions
   (which should, if there were any, be listed in the History section
   of the Document).  You may use the same title as a previous version
   if the original publisher of that version gives permission.
   
\item[B.]
   List on the Title Page, as authors, one or more persons or entities
   responsible for authorship of the modifications in the Modified
   Version, together with at least five of the principal authors of the
   Document (all of its principal authors, if it has fewer than five),
   unless they release you from this requirement.
   
\item[C.]
   State on the Title page the name of the publisher of the
   Modified Version, as the publisher.
   
\item[D.]
   Preserve all the copyright notices of the Document.
   
\item[E.]
   Add an appropriate copyright notice for your modifications
   adjacent to the other copyright notices.
   
\item[F.]
   Include, immediately after the copyright notices, a license notice
   giving the public permission to use the Modified Version under the
   terms of this License, in the form shown in the Addendum below.
   
\item[G.]
   Preserve in that license notice the full lists of Invariant Sections
   and required Cover Texts given in the Document's license notice.
   
\item[H.]
   Include an unaltered copy of this License.
   
\item[I.]
   Preserve the section Entitled ``History'', Preserve its Title, and add
   to it an item stating at least the title, year, new authors, and
   publisher of the Modified Version as given on the Title Page.  If
   there is no section Entitled ``History'' in the Document, create one
   stating the title, year, authors, and publisher of the Document as
   given on its Title Page, then add an item describing the Modified
   Version as stated in the previous sentence.
   
\item[J.]
   Preserve the network location, if any, given in the Document for
   public access to a Transparent copy of the Document, and likewise
   the network locations given in the Document for previous versions
   it was based on.  These may be placed in the ``History'' section.
   You may omit a network location for a work that was published at
   least four years before the Document itself, or if the original
   publisher of the version it refers to gives permission.
   
\item[K.]
   For any section Entitled ``Acknowledgements'' or ``Dedications'',
   Preserve the Title of the section, and preserve in the section all
   the substance and tone of each of the contributor acknowledgements
   and/or dedications given therein.
   
\item[L.]
   Preserve all the Invariant Sections of the Document,
   unaltered in their text and in their titles.  Section numbers
   or the equivalent are not considered part of the section titles.
   
\item[M.]
   Delete any section Entitled ``Endorsements''.  Such a section
   may not be included in the Modified Version.
   
\item[N.]
   Do not retitle any existing section to be Entitled ``Endorsements''
   or to conflict in title with any Invariant Section.
   
\item[O.]
   Preserve any Warranty Disclaimers.
\end{itemize}

If the Modified Version includes new front-matter sections or
appendices that qualify as Secondary Sections and contain no material
copied from the Document, you may at your option designate some or all
of these sections as invariant.  To do this, add their titles to the
list of Invariant Sections in the Modified Version's license notice.
These titles must be distinct from any other section titles.

You may add a section Entitled ``Endorsements'', provided it contains
nothing but endorsements of your Modified Version by various
parties--for example, statements of peer review or that the text has
been approved by an organization as the authoritative definition of a
standard.

You may add a passage of up to five words as a Front-Cover Text, and a
passage of up to 25 words as a Back-Cover Text, to the end of the list
of Cover Texts in the Modified Version.  Only one passage of
Front-Cover Text and one of Back-Cover Text may be added by (or
through arrangements made by) any one entity.  If the Document already
includes a cover text for the same cover, previously added by you or
by arrangement made by the same entity you are acting on behalf of,
you may not add another; but you may replace the old one, on explicit
permission from the previous publisher that added the old one.

The author(s) and publisher(s) of the Document do not by this License
give permission to use their names for publicity for or to assert or
imply endorsement of any Modified Version.


%\begin{center}
\subsection*{5. COMBINING DOCUMENTS}%WX%4|5. COMBINING DOCUMENTS|||
%\addcontentsline{toc}{section}{5. COMBINING DOCUMENTS}
%\end{center}


You may combine the Document with other documents released under this
License, under the terms defined in section 4 above for modified
versions, provided that you include in the combination all of the
Invariant Sections of all of the original documents, unmodified, and
list them all as Invariant Sections of your combined work in its
license notice, and that you preserve all their Warranty Disclaimers.

The combined work need only contain one copy of this License, and
multiple identical Invariant Sections may be replaced with a single
copy.  If there are multiple Invariant Sections with the same name but
different contents, make the title of each such section unique by
adding at the end of it, in parentheses, the name of the original
author or publisher of that section if known, or else a unique number.
Make the same adjustment to the section titles in the list of
Invariant Sections in the license notice of the combined work.

In the combination, you must combine any sections Entitled ``History''
in the various original documents, forming one section Entitled
``History''; likewise combine any sections Entitled ``Acknowledgements'',
and any sections Entitled ``Dedications''.  You must delete all sections
Entitled ``Endorsements''.

%\begin{center}
\subsection*{6. COLLECTIONS OF DOCUMENTS}%WX%4|6. COLLECTIONS OF DOCUMENTS|||
%\addcontentsline{toc}{section}{6. COLLECTIONS OF DOCUMENTS}
%\end{center}

You may make a collection consisting of the Document and other documents
released under this License, and replace the individual copies of this
License in the various documents with a single copy that is included in
the collection, provided that you follow the rules of this License for
verbatim copying of each of the documents in all other respects.

You may extract a single document from such a collection, and distribute
it individually under this License, provided you insert a copy of this
License into the extracted document, and follow this License in all
other respects regarding verbatim copying of that document.


%\begin{center}
\subsection*{7. AGGREGATION WITH INDEPENDENT WORKS}%WX%4|7. AGGREGATION WITH INDEPENDENT WORKS|||
%\addcontentsline{toc}{section}{7. AGGREGATION WITH INDEPENDENT WORKS}
%\end{center}


A compilation of the Document or its derivatives with other separate
and independent documents or works, in or on a volume of a storage or
distribution medium, is called an ``aggregate'' if the copyright
resulting from the compilation is not used to limit the legal rights
of the compilation's users beyond what the individual works permit.
When the Document is included in an aggregate, this License does not
apply to the other works in the aggregate which are not themselves
derivative works of the Document.

If the Cover Text requirement of section 3 is applicable to these
copies of the Document, then if the Document is less than one half of
the entire aggregate, the Document's Cover Texts may be placed on
covers that bracket the Document within the aggregate, or the
electronic equivalent of covers if the Document is in electronic form.
Otherwise they must appear on printed covers that bracket the whole
aggregate.


%\begin{center}
\subsection*{8. TRANSLATION}%WX%4|8. TRANSLATION|||
%\addcontentsline{toc}{section}{8. TRANSLATION}
%\end{center}


Translation is considered a kind of modification, so you may
distribute translations of the Document under the terms of section 4.
Replacing Invariant Sections with translations requires special
permission from their copyright holders, but you may include
translations of some or all Invariant Sections in addition to the
original versions of these Invariant Sections.  You may include a
translation of this License, and all the license notices in the
Document, and any Warranty Disclaimers, provided that you also include
the original English version of this License and the original versions
of those notices and disclaimers.  In case of a disagreement between
the translation and the original version of this License or a notice
or disclaimer, the original version will prevail.

If a section in the Document is Entitled ``Acknowledgements'',
``Dedications'', or ``History'', the requirement (section 4) to Preserve
its Title (section 1) will typically require changing the actual
title.


%\begin{center}
\subsection*{9. TERMINATION}%WX%4|9. TERMINATION|||
%\addcontentsline{toc}{section}{9. TERMINATION}
%\end{center}


You may not copy, modify, sublicense, or distribute the Document except
as expressly provided for under this License.  Any other attempt to
copy, modify, sublicense or distribute the Document is void, and will
automatically terminate your rights under this License.  However,
parties who have received copies, or rights, from you under this
License will not have their licenses terminated so long as such
parties remain in full compliance.


%\begin{center}
\subsection*{10. FUTURE REVISIONS OF THIS LICENSE}%WX%4|10. FUTURE REVISIONS OF THIS LICENSE|||
%\addcontentsline{toc}{section}{10. FUTURE REVISIONS OF THIS LICENSE}
%\end{center}


The Free Software Foundation may publish new, revised versions
of the GNU Free Documentation License from time to time.  Such new
versions will be similar in spirit to the present version, but may
differ in detail to address new problems or concerns.  See
\url{http://www.gnu.org/copyleft/}.

Each version of the License is given a distinguishing version number.
If the Document specifies that a particular numbered version of this
License ``or any later version'' applies to it, you have the option of
following the terms and conditions either of that specified version or
of any later version that has been published (not as a draft) by the
Free Software Foundation.  If the Document does not specify a version
number of this License, you may choose any version ever published (not
as a draft) by the Free Software Foundation.


%\begin{center}
\subsection*{ADDENDUM: How to use this License for your documents}%WX%4|ADDENDUM: How to use this License for your documents|||
%\addcontentsline{toc}{section}{ADDENDUM: How to use this License for your documents}
%\end{center}

To use this License in a document you have written, include a copy of
the License in the document and put the following copyright and
license notices just after the title page:

\bigskip
\begin{quote}
    Copyright \copyright  YEAR  YOUR NAME.
    Permission is granted to copy, distribute and/or modify this document
    under the terms of the GNU Free Documentation License, Version 1.2
    or any later version published by the Free Software Foundation;
    with no Invariant Sections, no Front-Cover Texts, and no Back-Cover Texts.
    A copy of the license is included in the section entitled ``GNU
    Free Documentation License''.
\end{quote}
\bigskip
    
If you have Invariant Sections, Front-Cover Texts and Back-Cover Texts,
replace the ``with...Texts.'' line with this:

\bigskip
\begin{quote}
    with the Invariant Sections being LIST THEIR TITLES, with the
    Front-Cover Texts being LIST, and with the Back-Cover Texts being LIST.
\end{quote}
\bigskip
    
If you have Invariant Sections without Cover Texts, or some other
combination of the three, merge those two alternatives to suit the
situation.

If your document contains nontrivial examples of program code, we
recommend releasing these examples in parallel under your choice of
free software license, such as the GNU General Public License,
to permit their use in free software.

%---------------------------------------------------------------------
%WX%3|GNU Free Documentation License||FDL_Section|


%% bibliography
%%WX%1|Bibliography|9000000000||
%\ifMKhtml
	%\clearpage
%\else
	%\cleardoublepage
	%\phantomsection
	%% why is this needed for pdf, but not htlatex (makes duplicate)?
	%\addcontentsline{toc}{chapter}{Bibliography}
%\fi
%\begin{thebibliography}{99}
	%\bibitem{bib-1}
		%Bib Item 1.
	%\bibitem{bib-2}
		%Bib Item 2.
%\end{thebibliography}


% index
%WX%1|Index|2000000000||
\ifMKhtml
	\clearpage
\else
	\cleardoublepage
	\phantomsection
	% why is this needed for pdf, but not htlatex (makes duplicate)?
	\addcontentsline{toc}{chapter}{Index}
\fi
\printindex

\end{document}                 % The input file ends with this command.

