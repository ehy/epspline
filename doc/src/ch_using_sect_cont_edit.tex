	An idea of a form will
	suggest where points should be
	placed, but usually the curve that appears is less than
	ideal when first created. In fact, placing
	\IXarg{points} with mouse clicks, as described
	so far, is just a rough start. If an idea of an
	object for a \IXpov{} \IXarg{scene} can be
	implemented as a \IXpov{} \IXprism{} or
	\IXlathe{}, then \IXpkg{} might help
	\emph{develop} the object. That is, making
	any but the simplest objects will be a process
	of changes and tuning. In this process, the
	interactive graphical interface should be
	an advantage. The use of the facilities
	provided by \IXpkg{} are discussed below.
	
		\subsection{Object Properties}%WX%4||||
		\label{ssec:object_props}
		To this point prisms and lathes have been
		referred to repeatedly, but there has been
		little, if any, mention of their properties.
		Moreover, it was stated that although an
		object's spline-type cannot be changed after
		it's created, it can be changed to a \IXprism{}
		or \IXlathe{}.
		
		The properties of objects in \IXpkg{}
		are changed in a \IXnewterm{dialog} window,
		``Spline Properties,'' which is invoked
		through an object \IXarg{menu}. This menu,
		seen in figure~\ref{fig:object_popup_menu},
		will `pop up' in response to a secondary
		button click on a selected object.
		(If no object is selected, the secondary
		button click will produce the drawing area
		menu.)
		
		\begin{figure}[htb!]
		\centering
		\includegraphics{\ImgCtxMenuB}
		\caption{The object ``pop-up'' menu.}
		\label{fig:object_popup_menu}
		\end{figure}

		The object menu has four items (after the label,
		``Selection''). The last three items are
		equivalent to similarly named items on the
		edit \IXarg{menu} (see \nameref{sssec:edit_menu}).
		The first item, as its name suggests, invokes
		the ``Spline Properties'' dialog, shown in
		figure~\ref{fig:props_dlg}.

		\begin{figure}[htb!]
		\centering
		\includegraphics{\ImgPropDlgA}
		\caption{The properties window.}
		\label{fig:props_dlg}
		\end{figure}
		
		The ``Spline Properties'' \IXarg{dialog} fields are as follows:
		\begin{description}
		  \item[(label)] A label stating the type of \IXspline{}.
		  \item[Object Name] A text entry field: a name
			for the object, suitable for \IXpov{} \IXarg{SDL},
			should be entered here before export. The text
			field will reject any characters that \IXpov{}
			does not accept in names, and if the name clashes
			with a \IXpov{} reserved word, a prompt for a
			new name is displayed.
		  \item[Render Type] A selection of rendering options:
			as \IXlathe{}, as \IXprism{}, or ``Undefined,''
			which means that it will not
			be rendered; this can be used to hide an object in
			a \IXpov{} \IXarg{scene} while keeping it in the
			\IXpkg{} file. This ``Undefined'' type must be given
			a name too, because it is exported to SDL as a \IXpov{}
			array (which might be useful).
		  \item[Sweep Type] This applies to prisms. The sweep
			type is described well in the \IXpov{}
			\href{\URLPOVdocsPrism}{\IXprism{} documentation}.
			Note that presently a conic sweep \IXprism{} is best
			if moved so that its center occurs at the origin
			of the \IXpkg{} drawing area.
		  \item[Use Sturm, Open] These are optional. Sturm selects
			a more accurate but slower algorithm; unless you
			know that you need this, you probably don't.
			Open applies to the \IXprism{}: the flat faces will not
			be rendered, leaving an open outline.
		  \item[Start, End Sweep] This applies to prisms:
			roughly, the offset and height of the \IXprism{}.
		  \item[Texture, Interior, Transform] Text entry
			fields that can accept a name for the attribute
			that must be
			`\verb!#declare!'ed
			before the exported \IXarg{SDL} is included
			in a scene. These fields appear with a default
			string (until changed) that is not applied to
			the object --- the default is only a place-holder.
			These attributes can probably be applied more
			flexibly in the using SDL.				
		  \item[Predefined Identifiers] This text entry field
			can accept more than one line. It is poorly named
			because it is not restricted to identifiers;
			for example, a comment may be added here.
			The most important point about this field is
			that it is not checked at all by \IXpkg{} ---
			the user must be certain that anything added here
			is valid SDL. The contents will be added near the
			end of the object definition.
		  \item[(buttons)] For ``OK'' and ``Cancel,''
			dismiss the dialog window, either
			effecting or cancelling any changes that were made.
			For ``Help,'' dismiss the dialog window with changes lost
			and display the help viewer showing this section.
		\end{description}

		Obviously, some object properties should be set
		before export to \IXarg{SDL}. In particular, each
		object should be given a name.

		\subsection{\dtypkgu{} Transforms and Copies}%WX%4||||
		\label{ssec:epspline_transforms}
		The transforms provided by \IXpov{} \IXarg{SDL}
		will be necessary in any any scene, and some
		that pertain to \IXpkg{} output are
		discussed in \nameref{sec:useful_transforms}.
		\IXpkgu{} provides some interactive transforms
		that should be useful while developing objects.
		These are applied directly to the control \IXarg{point}
		values rather than as exported SDL statements.
		
		Objects may be moved. Click an object to select it,
		and then \IXnewterm{drag} it with the mouse, and
		release it in the desired position. Single-pixel
		movements can be made with the arrow keys. Larger
		movements can be made with the page up and page down
		keys; if the control (or command) key is down then
		the page-key movements will be horizontal. Note that
		to move the whole object with keys, make sure that
		no control point is selected, or the point will be
		moved.
		
		Objects may be scaled, sheared, or rotated.
		To enter the mode in which these transforms are
		done, first select the object, then hold the
		shift \IXarg{key} down and click with the secondary
		mouse button. An object selected for these
		transforms is seen in figure~\ref{fig:sel_for_trans_a}.
		To leave this mode, another secondary click, but
		\emph{without} the shift key down, will return
		the object to the simple selected state.
		Repeated shift-key+secondary-clicks
		cycle through the scale, shear, and rotate modes,
		in order.

		\begin{figure}[htb!]
		\centering
		\includegraphics[width=\linewidth]{\ImgSelTransformA}
		\caption{Selected object in transform mode.}
		\label{fig:sel_for_trans_a}
		\end{figure}
		
		In transform mode small squares are
		shown just inside the object's bounding box, at the
		corners and at the middle of the sides. These
		squares are \IXnewterm{handles} for manipulating
		the object: \IXarg{drag} the handles to effect the
		transform.
		
		In figure~\ref{fig:sel_for_trans_a}
		two small concentric circles can be seen at the
		middle of the object. These circles are shown for
		shearing and rotation, and may be dragged with the
		mouse to a new location.
		The transform will be centered at the circles.
		They are not shown in the scaling mode, which uses no
		center as such.
		
		For each of the
		scale, shear, and rotate modes the shape of the
		mouse pointer will change. The pointer shapes
		should be suggestive of the mode (in some
		\IXarg{GTK} themes of the last several years the
		pointer shapes no longer seem to make sense, but the
		user should still form an association between the
		pointer shapes and the modes).
		
		Each transform, effected by dragging a handle with the
		mouse, is modified by one or more keys.	If the shift
		key is held down, the shear and rotate transforms
		will be constrained to a 15 degree angle (presently,
		this angle can only be changed as a compile-time
		macro, but may be made configurable
		in a future release if \IXpkg{} warrants more work).
		The scale mode transform, with the shift key, will become
		instead a \IXnewterm{flip} of the object horizontally,
		vertically, or diagonally according to the handle
		used. Also with the scale transform, if the `alt'
		key is held the transform will be directly proportional
		when using the upper-left or lower-right handle-squares,
		or inversely proportional with upper-right or
		lower-left handles. The alt key has no effect on
		scaling with the middle-of--the--sides handles, or on
		the shear and rotate transforms.
		
		A copy of the original, un-scaled object is left
		if the control (or command) key is down
		at the start of any transform. To make a copy that
		does not require a transform, simply make a
		\IXarg{clipboard} copy with the edit \IXarg{menu}
		(see \nameref{sssec:edit_menu}) or \IXarg{tool} bar
		(see \nameref{ssec:tool_bar}) and
		\IXarg{paste} the copy back. Whenever a copy is
		made, remember to make sure that each has a unique
		name (this is not done automatically, at present;
		see \nameref{ssec:object_props}).
		
		\subsection{Adding, Moving, and Deleting Points}%WX%4||||
		\label{ssec:add_del_points}
		Points can be added to an existing object
		for two purposes:
		as part of a new discontinuous sub-curve, or as
		part of an existing curve. The former case is
		discussed later in the
		section~\nameref{ssec:discont_objects}.
		
		To add a point to an existing curve, first make sure
		the object is selected. Place the mouse active point
		\emph{on} the curve at the desired position.
		Hold the shift \IXarg{key} down and click the
		primary mouse button. If the mouse pointer was
		properly on the curve, a new point will appear as a
		cyan colored square. If a new point appears as
		a red circle, the mouse active point did not lie on the
		curve, and the point should not be left in place.
		The errant new point can be removed with
		``Undo'' on the edit \IXarg{menu}
		(see \nameref{sssec:edit_menu}) or \IXarg{tool} bar
		(see \nameref{ssec:tool_bar}),
		or the \IXnewterm{control+z} key combination.
		
		A `point' is added to the \IXbezn{} curve in the
		same way, but the actual result is that the current
		segment is divided in two at the mouse point, and
		several necessary points are added. Some effort is
		made to place new \IXnewterm{tangent} control
		points sensibly, but adjustment will be needed.
		
		Moving points should be obvious: select the point so
		that it appears as a cyan colored square and drag it
		with the mouse. Only the \IXbezn{} curve needs more
		explanation. The section
		``\nameref{sssec:editing_points_bezier}'' explained
		that the last and first end-points of neighboring
		segments are (must be) coincident. \IXpkgu{}
		actually maintains an association of four points:
		the last and first \IXarg{tangent} control points
		and the end-points of neighboring segments. When any
		of these is selected the whole group will be
		colored cyan, and the selected point will be square.
		Within this group, dragging a \IXarg{tangent}
		control point will also move the other tangent
		point onto a line that passes through the endpoints;
		this produces a smooth curve through the end-points.
		A tangent point can be moved independently by
		pressing the control (or command) key, making
		an angle at the end-points. The end-points can
		be dragged, and they are always kept together
		as one. Dragging the end-points will also make
		an equivalent movement of the
		tangent points, causing a movement of the
		whole group; the end-points can be moved
		without affecting the tangent points by
		pressing the control (or command) key.
		
		Selected points may be deleted with the delete key
		or the edit \IXarg{menu} ``Delete'' item.
		In the case of the
		\IXbezn{} curve a whole segment is
		deleted, and remaining segments are joined at
		end-points. Adjustment will be needed.

		\subsection{Discontinuous Objects}%WX%4||||
		\label{ssec:discont_objects}
		Figure~\ref{fig:sel_for_trans_a} was presented to
		show a
		``\nameref{fig:sel_for_trans_a}.''
		It also shows a \IXspline{} object that is more
		elaborate than any in the earlier figures:
		the single object shows several closed sub-curves
		comprising the whole object.
		\IXpov{} handles this well, as
		seen in figure~\ref{fig:subcurves_1}
		(which is a \IXpov{} preview;
		see \nameref{sssec:help_menu}).

		\begin{figure}[htb!]
		\centering
		\includegraphics[width=\linewidth]{\ImgSubCurvesPreviewA}
		\caption{Preview image of a discontinuous object.}
		\label{fig:subcurves_1}
		\end{figure}
		
		Creating a \IXarg{sub-curve} in a
		\IXspline{} object is similar
		to the initial creation of a new \IXspline{} object.
		If necessary, review
		``\nameref{ssec:creating_newobj}.''
		A new \IXarg{sub-curve} cannot be started while
		the object is already in edit mode; if it is,
		then leave edit mode by depressing the \IXarg{shift}
		\IXarg{key} and clicking with the primary button.
		
		Select an existing object with a primary button
		click, and with the \IXarg{shift} \IXarg{key}
		depressed, another primary click will enter
		edit mode; unless the mouse pointer active point
		is very near an existing sub-curve, in which
		case a new point or segment is added to that
		sub-curve (use ``Undo'' if this happens).
		In edit mode now, points are added with clicks
		just as in the initial creation of an object,
		but they will compose a new \IXarg{sub-curve}.
		Add points just as described in
		``\nameref{ssec:creating_newobj},'' and make
		sure the \IXarg{sub-curve} is properly closed
		in the manner required for the type of \IXspline{}.
		Leave edit mode with shift-key+primary-click.
		
		A \IXarg{sub-curve} may be placed within another
		\IXarg{sub-curve}, in which case it will be
		rendered by \IXpov{} as a hole (and a
		\IXarg{sub-curve} placed within a ``hole''
		will be rendered as a `fill').
		A \IXarg{sub-curve} may be placed so that it
		is not within any other, in which case it will be
		rendered as if it were a distinct object
		(although it will still be part of the whole).
		Also, a \IXarg{sub-curve} may be placed such
		that its path crosses paths of other sub-curves,
		and \IXpov{} will render these with `holes'
		or `fills' at the intersections. Of course,
		all the above may be combined.

		\subsection{Guide Lines}%WX%4||||
		\label{ssec:guide_lines}
		\IXpkgu{} provides for simple \IXnewterm{guide} lines
		in the drawing area, which display as thin
		(one pixel wide) horizontal or vertical
		red lines. These are saved with \IXpkg{}'s
		files, so they are persistent, but they are
		not exported as \IXarg{SDL} and have no effect
		on the rendering of objects --- they are only
		tools for editing in \IXpkg{}.
		
		To add a guide line, place the mouse pointer
		over one of the graduated scales at the left
		and top of the drawing area
		(see ``\nameref{ssec:drawing_area}''),
		hold the shift key
		down, and `drag' with the mouse into the canvas
		(the work area with the blue grid). Do not release
		the mouse button until the new guide is at the
		desired position (the shift key may be released
		as soon as this action is initiated). A horizontal
		guide is produced from the top scale, and a vertical
		\IXarg{guide} from the left scale.
		
		If necessary, use the scroll bars to adjust the
		view of the canvas before adding a guide. Existing guides
		may be moved by dragging with the mouse. To
		pick up a guide, the mouse pointer active point
		must be placed directly over the line; since this
		can be difficult, it might help to place guides
		at incremental positions (e.g., a five pixel offset)
		so that the mouse point coordinates displayed on the
		status bar (see ``\nameref{ssec:status_bar}'')
		can assist in positioning the pointer at such
		intervals (irregular positions will be difficult
		to remember).
		
		To remove a \IXarg{guide} line, drag it back to the
		graduated scale and release it there.
		
		When moving a selected object or point, by default
		the \IXarg{guide} lines are merely a visual guide,
		but temporarily depressing the shift key will
		provide a light `snapping' to the guide when a
		point is in close proximity.

		%\subsection{Missing Features}%WX%4||||
		%\label{ssec:missing_todo}















