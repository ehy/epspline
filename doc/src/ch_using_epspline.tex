\chapter{Using \dtypkgu{}}

	\section{The Main Window Interface}%WX%3||||
	The elements of the \IXarg{interface} presented by \IXpkg{}
	have an arrangement that has been common in
	windowing systems for many years, and will probably
	seem familiar to most users. Most individual elements are common,
	but a few are specific to \IXpkg{}. A few are similar
	to common elements in most windowed applications,
	but with some difference in behavior.

	The main groups of \IXarg{interface} elements are
	discussed briefly, from the top down in order of appearance,
	in the following subsections. An image of the main window
	is shown in Figure~\ref{fig:bezier_linked_control_points}.

		\subsection{The Title Bar}%WX%4||||
		The \IXnewterm{title bar} is not actually part of
		the interface provided by an application. It is
		created and managed, if it exists at all, by a
		component of the windowing system. When present,
		it is usually similar for most applications. Some
		windowing systems (or system configurations) do not
		attach a \IXarg{title bar} to the application window,
		and may use another graphical device such as a
		single `bar' at the top of the screen that will
		change to represent any window that has the \emph{focus}.
		
		Regardless of the way in which a \IXarg{title bar}
		might be presented, an application such as \IXpkg{}
		has little control over it. The presence and position
		of buttons on the \IXarg{title bar}, for example,
		are a feature of the system. \IXpkgu{} only sets
		the text of the title.

		\subsection{The Menu Bar}%WX%4||||
		The \IXarg{menu} ``bar'' consists of the labels, arranged
		horizontally, nearest the top of the window. Each
		label can be selected with the \IXarg{mouse} and
		will produce a small rectangular window, the
		\IXnewterm{menu}, with selectable labels, or
		\emph{items}.
		
		Selecting a \IXarg{menu} item invokes an action
		such as saving or closing data. When an item does not
		make sense for the current state of the data, it
		cannot be selected, and this is indicated by showing the
		label ``greyed,'' or dimmed, in appearance.
		
		Many of the \IXarg{menu} labels have additional text,
		usually near the right-side edge, such as ``Ctrl+S.''  These
		indicate combinations of keys that will invoke the
		action associated with the \IXarg{menu} item without
		selecting the \IXarg{menu} with the mouse. These
		key combinations are often called ``accelerators.''
		
		In the following subsections on the menus, the \IXarg{menu}
		items are presented in tables with short descriptions of
		the associated actions. Items that differ from
		common use or are unique to \IXpkg{} are described in
		more detail following the table.
		Also, note that the accelerator keys that are shown
		may differ from those seen in a running instance of
		\IXpkg{}. The current windowing \IXarg{toolkit} or environment
		might provide accelerators for some common \IXarg{menu}
		items, and \IXpkg{} might use them.

			\subsubsection{The File Menu}%xxxWX%5||||
			This \IXarg{menu} provides items that generally
			apply to the current data, or ``file.''

			\begin{table}[htbp]
			\centering
			\textsf{
			\begin{tabular}[c]{ | l | l || p{5.33cm} | }
			\hline
			\textsc{Item}: & \textsc{Keys}: & \textsc{Action}:\\ \hline
			\hline \hline
			Save & Ctrl+S & save data with current name\\ \hline
			Save As & --- & save data with a new name\\ \hline
			New & Ctrl+N & create a tab for new data\\ \hline
			Open Here & --- & open (load) file data in current tab\\ \hline
			Open & Ctrl+O & open (load) file data in a new tab\\ \hline
			Close & Ctrl+W & close the data (file), leave tab\\ \hline
			Close Tab & --- & close the data (file), close tab\\ \hline
			Export & Ctrl+E & save the data as \IXpov{} \IXarg{SDL}\\ \hline
			Export As & --- & save as \IXarg{SDL} with a new name\\ \hline
			\hline
			Quit & Ctrl+Q & quit the \textrm{\dtythispkgname{}} program\\ \hline
			\end{tabular}
			}%\textsf
			\caption{The Items of the File Menu}
			\label{tab:File_menu}
			\end{table}

			The unusual items that need more explanation are:
			\begin{description}
			  \item[New] The action is to create a new tab in the
			  tabbed interface, without contents, suitable for
			  creating a new work. Alternatively, a file can be opened
			  in the new empty tab with ``Open Here.''
			  \item[Open Here] Open a file and load the data into
			  the current tab. If the tab contains data with
			  unsaved changes the user will be asked whether to
			  proceed (which will close the current file,
			  discarding changes) or cancel.
			  \item[Close Tab] This will close the file data
			  \emph{and} the tab, while the item ``Close''
			  will only close the file data, and leave the tab
			  open and empty. If there are unsaved changes the
			  user will be prompted.
			  \item[Export] \IXpkgu{} saves files in its own
			  simple format; to use the data with \IXpov{}
			  it must be ``exported'' to \IXarg{SDL}. This item
			  will export the data to a file with an existing
			  name that had be given with ``Export As.''
			  \item[Export As] Like ``Export'' but shows a file
			  selector dialog window so that the user may provide
			  a file name. This is necessary the first time the
			  file is exported; thereafter ``Export'' can be used
			  to overwrite the current file with changes. (Note that
			  the file name suffix `.inc' is commonly used for
			  such \IXarg{SDL} fragments, but it is not required by
			  \IXpov{} or appended by \IXpkg.)
			\end{description}

			\subsubsection{The Edit Menu}%xxxWX%5||||
			This \IXarg{menu} provides items that
			affect or change the current data.

			\begin{table}[htbp]
			\centering
			\textsf{
			\begin{tabular}[c]{ | l | l || p{5.33cm} | }
			\hline
			\textsc{Item}: & \textsc{Keys}: & \textsc{Action}:\\ \hline
			\hline \hline
			Undo & Ctrl+Z & revert the last change made\\ \hline
			Redo & Ctrl+R & do again a change that was reverted\\ \hline
			\hline
			Copy & Ctrl+C & copy the selected object onto internal ``clipboard''\\ \hline
			Cut & Ctrl+X & remove, after copying the selected object onto internal ``clipboard''\\ \hline
			Delete & --- & remove selected object\\ \hline
			Paste & Ctrl+V & place an object from internal ``clipboard'' in the data\\ \hline
			Copy Global & --- & like ``Copy'', onto a ``clipboard'' available to all tabs\\ \hline
			Cut Global & --- & like ``Cut'', onto a ``clipboard'' available to all tabs\\ \hline
			Paste Global & --- & like ``Paste'', from a ``clipboard'' available to all tabs\\ \hline
			\hline
			Down & --- & move the selected object down in $z$-order\\ \hline
			Up & --- & move the selected object up in $z$-order\\ \hline
			\end{tabular}
			}%\textsf
			\caption{The Items of the Edit Menu}
			\label{tab:Edit_menu}
			\end{table}

			The unusual items that need more explanation are:
			\begin{description}
			  \item[Copy, Cut, Paste] These item names are common
			  in windowing system programs, but in \IXpkg{} data
			  is \emph{not} placed on a system ``clipboard,'' but
			  is only available within \IXpkg. Moreover, data
			  copied or cut by these items can only be pasted into
			  the same tab.
			  \item[Copy, Cut, Paste --- Global] These differ from
			  the items without ``Global'' only in that data
			  copied or cut from one tab is available to be pasted
			  into the data of another tab.
			  \item[Down, Up] The drawing area may have several
			  objects, and these are kept in a stacking order, or
			  ``$z$-order.'' This may not be apparent in the
			  two-dimensional interface, but will be noticed when
			  the user attempts to select an object that is below
			  another. In such a case, it might be necessary
			  to move the obstructing object down in the order
			  to make the wanted object selectable. Note that
			  there is one $z$-order for all objects, even
			  those that don't overlap, so it might be necessary
			  to select ``Down'' several times (and on more
			  than one object) before the wanted
			  object is no longer obstructed. If the wanted object
			  can be selected, but only around the edges of an
			  obstructing object, then ``Up'' can be used to
			  raise the selection and make it easier to select
			  with the next attempt. Note also that the
			  stacking or $z$-order has no effect on the way
			  \IXpov{} will render the objects, and will only affect
			  the order in which the objects appear in exported
			  \IXarg{SDL}.
			\end{description}
			
			A final note on the ``clipboard'': it's an omission
			that the system clipboard is not used. It would be
			useful if a selection could be copied (or cut) to
			the system clipboard as a \IXarg{SDL} fragment
			which could be pasted into a text editor. Likewise,
			an attempt could be made to paste from the system
			clipboard, if in text form, by trying to parse a
			prism or lathe definition in \IXarg{SDL}. This
			feature might be added in the future if \IXpkg{}
			warrants more work.

			\subsubsection{The Tools Menu}%xxxWX%5||||
			This \IXarg{menu} has miscellaneous items that
			are specific to \IXpkg.

			\begin{table}[htbp]
			\centering
			\textsf{
			\begin{tabular}[c]{ | l | l || p{5.33cm} | }
			\hline
			\textsc{Item}: & \textsc{Keys}: & \textsc{Action}:\\ \hline
			\hline \hline
			Set Scale & --- & show a dialog to set arbitrary scale of display\\ \hline
			Cycle Scale & --- & cycle scale of display through normal, large, and small\\ \hline
			Toggle Quick Drawing & --- & switch between anti-aliased, and not, outlines\\ \hline
			Save Visible to File & --- & save the visible part of drawing area to an image file\\ \hline
			\end{tabular}
			}%\textsf
			\caption{The Items of the Tools Menu}
			\label{tab:Tools_menu}
			\end{table}

			The Tools \IXarg{menu} items all need explanation:
			\begin{description}
			  \item[Set Scale] The view of the drawing area can be
			  displayed with a scaling factor. This feature uses a
			  capability provided by the \IXarg{\emph{wxWidgets}}
			  library; that is, \IXpkg{} is not drawing lines at
			  different scales, but the view in the window is scaled.
			  Therefore, with a large scale lines (and all drawn
			  items) will appear thicker, and details such as
			  anti-aliasing will become apparent. With a small scale,
			  some lines of one-pixel--width are lost. Nevertheless,
			  this feature can be useful at times. Generally, most
			  work will be easiest at 100\% (normal) scale.
			  \item[Cycle Scale] This is a quick way to rotate through
			  100\%, 200\%, and 50\% scales, and is probably more
			  generally useful than ``Set Scale.''
			  \item[Toggle Quick Drawing] The object outlines, the
			  curves, are drawn by default with anti-aliasing. The
			  alternative drawing method, the ``quick'' method, simply
			  uses several simple lines drawn with
			  \IXarg{\emph{wxWidgets}}' graphics \IXnewterm{API} calls
			  to approximate the curve forms. The latter is not
			  likely to be useful unless the drawing area is on
			  a display to which raster image data transfers are
			  slow. Most modern computers are much faster than
			  necessary for the anti-aliased drawing if the the
			  display is on the same machine that is running \IXpkg.
			  If the display is not on the same host that is running
			  \IXpkg, the anti-aliased drawing might be too slow to
			  be usable (particularly over a slow link such as
			  802.11). That is because the anti-aliasing must be
			  done on a raster image, and the whole image,
			  which is a large chunk of data, transferred to the display.
			  With ``quick drawing'' the \IXarg{API} line drawing
			  is accomplished with small data transfers, and will
			  be quicker, and probably usable (but less accurate
			  in appearance).
			  \item[Save Visible to File] The part of the drawing area
			  that is visible within the \emph{scrolled} window can
			  be saved to a raster image file. This is (the remains
			  of) a development feature, and as it is difficult to
			  imagine how it might be useful to users, it is likely
			  to be removed at some point. It remains, for now,
			  mostly because it is harmless and adds very little
			  size.
			\end{description}
			
			Note on ``Toggle Quick Drawing'': above, it was said
			that anti-aliased drawing is the default. There is an
			exception under the \IXarg{\emph{X Window System}},
			where the ``DISPLAY'' environment variable is checked
			for a host component, i.e., ``somehost:0.0'' rather
			than simply ``:0.0''. If a host part is found, even
			if it is \emph{localhost}, the default will be switched
			to ``quick'' (not anti-aliased) drawing, on the obvious
			assumption that the display is remote (or remote-like).
			Of course, the user may still select
			``Toggle Quick Drawing'' and decide whether performance
			with ant-aliasing is acceptable.
			
			\subsubsection{The Help Menu}%xxxWX%5||||
			This \IXarg{menu} appears in most windowing system
			applications, and provides items that might
			be helpful in some way.

			\begin{table}[htbp]
			\centering
			\textsf{
			\begin{tabular}[c]{ | l | l || p{5.33cm} | }
			\hline
			\textsc{Item}: & \textsc{Keys}: & \textsc{Action}:\\ \hline
			\hline \hline
			Help & Ctrl+H & show the \IXpkg{} online help\\ \hline
			\hline
			Preview & --- & invoke \IXpov{} with temporary \IXarg{SDL} to view the work\\ \hline
			\hline
			About & --- & show the ``About'' window\\ \hline
			\end{tabular}
			}%\textsf
			\caption{The Items of the Help Menu}
			\label{tab:Help_menu}
			\end{table}

			Additional explanation:
			\begin{description}
			  \item[Help] This will show (or bring to the
			  forefront, if already shown) a \IXnewterm{help viewer}
			  window with the ``Contents'' Chapter visible.
			  (The \IXarg{help viewer} is a non-trivial piece of
			  software provided entirely by \IXarg{\emph{wxWidgets}},
			  and much to its credit --- developers take note.)
			  \item[Preview] The \IXpov{} preview does not attempt
			  to make a very nice rendering. It does attempt to
			  make an image with a perspective that includes all
			  objects. Only a few simple colors are used. Although
			  crude, the preview will often be useful. An important
			  thing to note is that \IXpkg{} does not delete the
			  temporary \IXarg{SDL} files and output until the
			  running \IXpov{} has been quit. If \IXpkg{} is
			  quit while a \IXpov{} child is still running, temporary
			  files are left in place.
			  \item[About] The ``About'' window includes copyright
			  and license information.
			\end{description}
			
			A further note on the preview: \IXpov{} for Unix-like
			systems is based on a command-line interface, but can
			also draw in real-time to a simple \IXnewterm{X}
			window, and this suits the preview well. \IXpov{}
			will quit with a single click on its window, or the
			\emph{Q} key. \IXpov{} for Microsoft Windows is quite
			different. That version will show an interactive control
			and editing window before the real-time display window,
			and might prompt the user with several dialog windows
			too (and might even play a little music). It is a nice
			tool for developing \IXpov{} images, but it is a little
			awkward when used for a simple preview by another
			program. Nevertheless,
			the preview remains useful. The Microsoft Windows version
			of \IXpov{} is quit with a menu or button; not a click.
			The same caveat about quiting \IXpkg{} with
			a child \IXpov{} instance running applies.


		\subsection{The Tool Bar}%WX%4||||
		The \IXnewterm{tool} bar just below the \IXarg{menu}
		bar. It is a row of buttons with icons that can be
		clicked to invoke actions. Each action available on
		the \IXarg{tool} bar is equivalent to a \IXarg{menu} item
		action. Naturally, the \IXarg{tool} bar will be more
		convenient while working than the equivalent \IXarg{menu}
		items. From left to right, the \IXarg{tool} bar
		items are:

			\begin{description}
			  \item[Quit] Same as ``Quit'' in the
			  ``File'' \IXarg{menu}.
			  \item[Open New Tab] Same as ``New'' in the
			  ``File'' \IXarg{menu}.
			  \item[Open] Same as ``Open'' in the
			  ``File'' \IXarg{menu}.
			  \item[Save] Same as ``Save'' in the
			  ``File'' \IXarg{menu}.
			  \item[Paste] Same as ``Paste'' in the
			  ``Edit'' \IXarg{menu}.
			  \item[Copy] Same as ``Copy'' in the
			  ``Edit'' \IXarg{menu}.
			  \item[Cut] Same as ``Cut'' in the
			  ``Edit'' \IXarg{menu}.
			  \item[Move Down] Same as ``Down'' in the
			  ``Edit'' \IXarg{menu}.
			  \item[Move Up] Same as ``Up'' in the
			  ``Edit'' \IXarg{menu}.
			  \item[Undo] Same as ``Undo'' in the
			  ``Edit'' \IXarg{menu}.
			  \item[Redo] Same as ``Redo'' in the
			  ``Edit'' \IXarg{menu}.
			  \item[Cycle Scale] Same as ``Cycle Scale'' in the
			  ``Tools'' \IXarg{menu}.
			  \item[Preview] Same as ``Preview'' in the
			  ``Help'' \IXarg{menu}.
			  \item[Show Application Help] Same as ``Help'' in the
			  ``Help'' \IXarg{menu}.
			\end{description}
			
		\begin{figure}[htbp]
		\centering
		\includegraphics[width=\linewidth]{\ImgUsingA}
		\caption{\dtypkgu{}'s tool bar.}
		\label{fig:tool_bar_1}
		\end{figure}

		Figure~\ref{fig:tool_bar_1} shows the \IXarg{tool}
		bar with its buttons. The icons shown are from a
		\IXnewterm{desktop} \emph{theme}, and may be
		different for each theme, in the \IXarg{GTK2} version
		of \IXpkg{}. In other versions a set of icons included
		with \IXarg{\emph{wxWidgets}} is used.
		Depending on the current platform, it might be possible
		to grab the \IXarg{tool} bar and drag it off the
		main window, and, of course, back onto the window.


		\subsection{The Tab Selection Area}%WX%4||||
		The \IXnewterm{tab} area is between the \IXarg{tool}
		bar and the \IXarg{drawing area}. There may be one or
		several \IXarg{tabs} present. \IXpkgu{} uses tabs to
		allow holding several sets of data, or files, open at
		once. Each \IXarg{tab} functions like a button: when
		the tab is clicked the associated data is brought to the
		front (i.e., shown). Each tab has a label with the name
		of its file, or symbolic text meant to suggest that the
		data has not been saved to a file yet (or the tab is empty).
		
		As of \IXarg{\emph{wxWidgets}} version 2.8 two forms
		of tabs are available: a simple type available in previous
		versions, and a new type with more features. The simple
		type does not provide a close button on the \IXarg{tabs},
		and the order cannot be rearranged. The new type does
		provide for close buttons, and a \IXarg{tab} can be
		dragged with the mouse to a new position among the others.
		The new type also provides for a button at the right
		side of the \IXarg{tab} area, which produces a \IXarg{menu}
		of the available tabs for direct selection; this is
		convenient when there are many \IXarg{tabs} and some
		cannot be seen due to lack of space.
		
		\IXpkgu{} uses the new type of \IXarg{tabs} by default
		(except when built with \IXarg{\emph{wxWidgets}} 2.6).
		There is little reason to use the simple type, and to
		do so requires setting a condition when the program is
		built from source (which is left as an exercise for those
		so inclined).
		

		\subsection{The Drawing Area and Canvas}%WX%4||||
		This is where the work is done. At the top and
		to the left are graduated scales, and as the mouse
		is moved in the \IXnewterm{drawing} area a thin line
		is drawn on the scales to help locate the mouse active
		point with precision. To the right and at the bottom
		are scroll bars to move the visible part of the
		\IXarg{drawing} area.
		
		The `real' drawing area will be called the
		\IXnewterm{canvas}.
		The \IXarg{canvas} has a white background with a grid
		of thin blue lines. This is meant to suggest physical
		\IXnewterm{graphing paper}. (In the future there
		may be an option to disable the grid; there is none now).


		\subsection{The Status Bar}%WX%4||||
		The \IXnewterm{status} bar is at the bottom of
		the main window. It accepts no input; it exists
		only to provide information.
		
		The \IXarg{status}
		bar has three sections, or ``panes.'' The left, and
		longest, pane shows details about a selected object,
		such as its name and type, and if it is a prism, the
		type and extent of its \IXnewterm{sweep}. The left
		pane also shows temporary help messages associated
		with \IXarg{tool} bar or \IXarg{menu} items. Such help
		messages are shown when the mouse is over the item, and
		any original text is restored when the mouse leaves.
		The pane in the middle shows the coordinates of the
		mouse pointer (the pointer's active spot). The right
		pane shows the scale at which the drawing area is displayed.
		Also, if there are unsaved changes to the data, the
		right pane will show an asterisk.


	\section{Creating and Editing}%WX%3||||


		\subsection{Starting a New File}%WX%4||||

	\section{Tips}%WX%3||||


%\IXpkgu{} is an interactive program with a
%graphical interface. Interaction is primarily by use of a mouse
%(or any pointing device from which the windowing system delivers
%`mouse' events to software).
%\subsection{Overview}%WX%4||||
%\subsection{Mouse and Key}%WX%4||||
%% no coded comment for subsubsection
%\subsubsection{Operations}%xxxWX%5||||
%\subsection{Usage Table}%WX%4||||
%
