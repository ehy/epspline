	\IXpkgu{} 
	The elements of the \IXarg{interface} presented by \IXpkg{}
	have an arrangement that has been common in
	windowing systems for many years, and will probably
	seem familiar to most users. Most individual elements are common,
	but a few are specific to A few are similar
	to common elements in most windowed applications,
	but with some difference in behavior.

	The main groups of \IXarg{interface} elements are
	discussed briefly, from the top down in order of appearance,
	in the following subsections. An image of the main window
	is shown in figure~\ref{fig:bezier_linked_control_points}.

		\subsection{Global Preferences}%WX%4||||
		\label{ssec:prefs_global}
		The \IXnewterm{title bar} is not actually part of
		the interface provided by an application. It is
		created and managed, if it exists at all, by a
		component of the windowing system. When present,
		it is usually similar for most applications. Some
		windowing systems (or system configurations) do not
		attach a \IXarg{title bar} to the application window,
		and may use another graphical device such as a
		single `bar' at the top of the screen that will
		change to represent any window that has the \emph{focus}.
		
		Regardless of the way in which a \IXarg{title bar}
		might be presented, an application such as \IXpkg{}
		has little control over it. The presence and position
		of buttons on the \IXarg{title bar}, for example,
		are a feature of the system. \IXpkgu{} only sets
		the text of the title.

		\subsection{POV-Ray Settings}%WX%4||||
		\label{ssec:prefs_povray}
		The \IXarg{menu} ``bar'' consists of the labels, arranged
