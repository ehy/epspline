	\IXpkgu{} can display a raster image on the drawing area.
	A \IXnewterm{background image} might be helpful for tracing,
	inspiration, or anything that pleases the user. \IXpkgu{}
	provides a \IXarg{dialog} window with which
	a \IXarg{background image} may be selected,
	adjusted and placed. This dialog window is
	invoked by selecting the tools \IXarg{menu}
	(see \nameref{sssec:tools_menu})
	``Set Background Image'' item. An existing image
	may be removed with the ``Remove Background Image''
	item on the same menu. Also, those two menu items
	appear near the bottom of the drawing area 
	``pop-up'' menu, seen in figure~\ref{fig:canvas_popup_menu}.

	\begin{figure}[htb!]
	\centering
	\includegraphics{\ImgBGImgDlg}
	\caption{The background image window.}
	\label{fig:bgimg_dlg}
	\end{figure}

	The \IXarg{background image} \IXarg{dialog} window is
	seen in figure~\ref{fig:bgimg_dlg}.
	There are five buttons near the bottom of the
	\IXarg{dialog} window. These are explained
	in the table \nameref{tab:bgimg_dialog_buttons}.
	Note that
	the buttons listed in the table as ``Help,'' ``Apply,'' ``Cancel,''
	and ``OK,'' might have different labels and appear in a
	different order depending on the current platform or theme
	(or translation).
	
	\begin{table}[htbp]
	\centering
	\textsf{
	\begin{tabular}[c]{ | l || p{6.33cm} | }
	\hline
	\textsc{Button}: & \textsc{Action}:\\ \hline
	\hline \hline
	Help & show the help viewer window displaying this section\\ \hline
	Apply & put current values in effect, and leave the dialog window showing\\ \hline
	Cancel & dismiss the dialog window, and discard changed values\\ \hline
	OK & put current values in effect, dismiss the dialog window\\ \hline
	\end{tabular}
	}%\textsf
	\caption{The Buttons of the Background Image Dialog Window}
	\label{tab:bgimg_dialog_buttons}
	\end{table}

	Like the
	\hyperref[sec:ch_global_prefs]{preferences} \IXarg{dialog},
	this \IXarg{dialog} window is designed to allow interaction
	with the
	\hyperref[sec:window_interface]{main window}
	without closing the \IXarg{dialog} window.

		\subsection{Background Image Dialog Controls}%WX%4||||
		\label{ssec:bg_image_controls}
		In addition to the buttons described above,
		the \IXarg{background image} \IXarg{dialog} window
		presents items to control the placement and
		appearance of the image, and how or whether
		the image will be saved when the \IXpkg{} file
		is saved. Hopefully, some combination of the
		controls will provide a suitable display on the
		drawing area.

		\begin{description}
		  \item[Copy options:] This exclusive option
		  has three choices.
			\begin{enumerate}
			  \item \emph{Copy Original}: A
			  copy of the image will be made when the
			  \IXpkg{} file is saved. The copy is made in
			  the same directory as the saved \IXpkg{} file,
			  and will have the same file name as the original,
			  but with a
			  `.png' suffix (as that format is used).  The
			  image is not changed, but changes applied on the
			  drawing area representation are preserved in
			  the \IXpkg{} file, and will be applied again when
			  the \IXpkg{} file is opened again.
			  \item \emph{Copy Changes}: This
			  choice will save a `.png'
			  file as above, but with changes applied to
			  the saved image. In this case the image
			  file will have the same name as the \IXpkg{} file,
			  but with a `.png' suffix. The \IXarg{dialog}
			  control settings, except the offsets,
			  are discarded so that they will not be applied
			  again in addition to the first application
			  in the saved copy. Offset settings are retained,
			  because they apply to the drawing area rather
			  than the image data.
			  \item \emph{No Copy}: This choice
			  saves no image file, but if the
			  original is found when the \IXpkg{} file is
			  opened again it will be used. Note that the
			  full path of the original file will be stored
			  in the \IXpkg{} file. If the path is unsuitable
			  for a file that might be transferred or shared,
			  then save a copy with the background image
			  removed and use that as a portable version.
			\end{enumerate}
		  \item[Greyscale:] This checkbox (on or off)
		  will apply a conversion to greyscale on a color
		  image.
		  \item[Flip Horizontal, Flip Vertical:] These checkboxes
		  apply a `mirror' operation, or `flip,' along one
		  axis. The horizontal operation appears to `flip'
		  the image around the $y$ axis,
		  and the vertical around $x$.
		  \item[Rotate:] A rotation is applied around the
		  center of the image. The value is in \IXnewterm{degrees}.
		  Note that the image will appear to become smaller,
		  because the original rectangular dimensions
		  are preserved. If necessary, compensate for that
		  with width and height adjustments.
		  \item[Width, Height:] These `spinner' controls accept
		  numeric (integer) input from keys, or mouse adjustment
		  by clicking the arrows at the right of the control.
		  In either of these two, a value of zero (0)
		  indicates that the the original value from the image
		  should be used.
		  \item[Horizontal Offset, Vertical Offset:] These
		  `spinner' controls accept positive or negative
		  integers. These values control placement of the image
		  within the drawing area. If these are both
		  zero (0) then the upper left corner of the image
		  is place at the upper left corner of the drawing
		  area.
		  \item[HSV Adjust:] These `slider' controls will
		  adjust some image color properties in the
		  \IXnewterm{HSV}
		  (``Hue, Saturation \& Value'') color space.
		  It might be possible to find a set of adjustments
		  with these controls that help emphasize an area
		  of interest in the image.
		  \item[Lightness:] This `slider' applies a simple
		  adjustment to apparently lighten or darken the
		  image. Using this to lighten an image should be
		  helpful --- and possibly necessary --- to display
		  a background image with sufficient contrast
		  against the \IXarg{curves} being drawn.
		  \item[Choose an image file:] This control has a
		  text entry field that will accept am image
		  file path, and a button that will invoke a
		  file selector \IXarg{dialog} window.
		\end{description}

		Many images will be too dark or colorful to serve
		as a background over which \IXarg{curves} can be
		seen and manipulated. The controls in this \IXarg{dialog}
		should make it possible to render an image as a
		useful background.
