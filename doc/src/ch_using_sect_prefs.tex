	\IXpkgu{} provides a \IXarg{dialog} window with which
	a set of \IXarg{preferences} can be set. This dialog window is
	invoked by selecting the edit menu
	(see \nameref{sssec:edit_menu}), ``Preferences'' item.
	The dialog window uses a \IXnewterm{tabbed} interface, and
	each \IXarg{tab} contains related settings.

	Five buttons near the bottom of the \IXarg{dialog} window
	affect the preferences and the window. These are explained
	in the table \nameref{tab:prefs_dialog_buttons}. Note that
	the buttons listed in the table as ``Help,'' ``Apply,'' ``Cancel,''
	and ``OK,'' might have different lables and appear in a
	different order depending on the current platform or theme
	(or translation).
	
	\begin{table}[htbp]
	\centering
	\textsf{
	\begin{tabular}[c]{ | l || p{6.33cm} | }
	\hline
	\textsc{Button}: & \textsc{Effect}:\\ \hline
	\hline \hline
	Restore Defaults & set \IXpkg{} defaults in all items\\ \hline
	Restore Configured & set values from stored configuration in all items\\ \hline
	\hline
	Help & show the help viewer window displaying this section\\ \hline
	Apply & put current values in effect, and leave the dialog window showing\\ \hline
	Cancel & dismiss the dialog window, and discard changed values\\ \hline
	OK & put current values in effect, dismiss the dialog window, and save the values in stored configuration\\ \hline
	\end{tabular}
	}%\textsf
	\caption{The Buttons of the Preferences Dialog Window}
	\label{tab:prefs_dialog_buttons}
	\end{table}

	The \IXarg{dialog} window is designed to allow interaction
	with the main window (see \nameref{sec:window_interface})
	while the \IXarg{dialog} window remains present. (The
	system might force the \IXarg{dialog} to remain above
	the main window, but it can be moved aside.) Changes to settings
	can be tried and changed again without dismissing the dialog
	and invoking it again from the \IXarg{menu} using the ``Apply''
	button. Finally, affect or discard changes with ``OK''
	or ``Cancel.''

		\subsection{Global Preferences}%WX%4||||
		\label{ssec:prefs_global}
		The \IXnewterm{title bar} is not actually part of
		the interface provided by an application. It is
		created and managed, if it exists at all, by a
		component of the windowing system. When present,
		it is usually similar for most applications. Some
		windowing systems (or system configurations) do not
		attach a \IXarg{title bar} to the application window,
		and may use another graphical device such as a
		single `bar' at the top of the screen that will
		change to represent any window that has the \emph{focus}.
		
		Regardless of the way in which a \IXarg{title bar}
		might be presented, an application such as \IXpkg{}
		has little control over it. The presence and position
		of buttons on the \IXarg{title bar}, for example,
		are a feature of the system. \IXpkgu{} only sets
		the text of the title.

		\subsection{POV-Ray Settings}%WX%4||||
		\label{ssec:prefs_povray}
		The \IXarg{menu} ``bar'' consists of the labels, arranged
