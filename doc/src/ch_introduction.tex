\chapter{Introduction}
\ifMKwx
	\small
\begin{verbatim}
Q: Why is it that the more accuracy you demand from an interpolation
   function, the more expensive it becomes to compute?
A: That's the Law of Spline Demand.
   [A Unix fortune.]
\end{verbatim}
	%\normalfont
	\normalsize
\else
	\epigraphhead[70]{
		\epigraph{
		\textit{Q: Why is it that the more accuracy you demand
			from an interpolation
		   function, the more expensive it becomes to compute?}
		
		\textit{A: That's the Law of Spline Demand.}}%
		{\textsc{A Unix fortune.}}%
	}% \epigraphhead[70]
\fi % \ifMKwx

% Introduction opening section
\section{The Software}%WX%3|||SoftwareIntroSection|
\IXpkgu{} is a simple, single-purpose program that might
be helpful to users of the \IXpov{} program.
To introduce \IXpkg{} a brief introduction to \IXpov{} is necessary.

	%WX%4|<r>Brief[ 	][ 	]*Introduction[ 	][ 	]*to[ 	].*POV-Ray||POVRayIntroSection|
	\subsection{Brief Introduction to \dtypov}
\IXpov{} is a \IXnewterm{ray-tracing} program with a long history.
Ray-tracing is a method for rendering
graphical images with a model of \IXnewterm{optics}.
Put simply, the user describes objects in space, a viewing position,
and sources of light. The program `traces' the path of a ray of
light from the view, among the objects in space, and back to the sources
of light. The method is repeated for each pixel of the image that
is being generated, and the pixels are colored according to the
interaction of the light and the objects in space that it encounters
(as seen from the view defined by the user).
The result of this method can be very impressive, and with its
long history of active development \IXpov{} does it very well.

The user of a ray-tracing program must describe the objects in space
and their attributes. For that purpose \IXpov{} provides a
\IXnewterm{scene description language}, or `\IXarg{SDL}.'
The \IXarg{SDL}
allows users to describe objects in numerous ways, all of which
are explained in detail and with examples in \IXpov{}'s
excellent \href{http://www.povray.org/documentation/}{documentation}
(\url{http://www.povray.org/documentation/}).
Many of the objects that can be defined in the \IXarg{SDL} are easily
edited (i.e., written) by hand, but some types of objects
are more suited to generation by an interactive program that
allows visual editing and feedback. \IXpov{} is not interactive;
it works with prepared input files. Fortunately, there are several
third-party programs available for interactively editing
\IXpov{} \IXarg{SDL} objects. Such programs typically
produce \IXarg{include} files with object definitions
that the the user will then refer to and arrange in a main scene file.

\IXpov{} is available cost-free as source code---and
ready-to--run binaries for some platforms---at
\IXpov's \href{http://www.povray.org/}{web site}
(\url{http://www.povray.org/}).
\footnote{Please note that, although the source code is available,
it is \emph{not} re-distributable in the manner allowed
by \emph{free software} or \emph{open source} licenses.
It is important that software licenses be respected.
The \IXpov{} license for personal use is at
\url{http://www.povray.org/povlegal.html}, and for distribution
at \url{http://www.povray.org/distribution-license.html}.}

\begin{figure}[htb]
\centering
\includegraphics[width=\linewidth]{\ImgIntroA}
\caption{An image rendered with \dtypov.}
\label{fig:pov_image_Intro_0}
\end{figure}


	%WX%4|<R>Introducing .*Epspline||EpsplineIntroSection|
	\subsection{Introducing \dtypkgu}
Two of the object types that may be defined in \IXpov's
\IXarg{SDL} are the `prism' and the `lathe,' and they are
the ``single-purpose'' of \IXpkg.
Figure~\ref{fig:pov_image_Intro_0} is a sample scene
composed mostly of \IXpov's prism and lathe objects
(there is also a `plane' to provide a background, and of course
light sources and a `camera', which is the point-of--view).

\begin{figure}[htb]
\centering
\includegraphics[width=\linewidth]{\ImgIntroB}
\caption{\dtybeznu{} spline with linked control points selected.}
\label{fig:bezier_linked_control_points}
\end{figure}

In the \IXpov{} scene description
language these objects are introduced
with the key words ``lathe'' and ``prism''.
The \IXpov{} documentation
describes these, and includes an example of creating such an object
by hand by carefully placing the control points.  The simplest of these
objects may well be manageable by hand, but if the object is not
simple it might have hundreds, even thousands, of control points.
It would be difficult to create and maintain such an object by hand,
and surely there few people who can visualize the resulting curve's
path.  Therefore an interactive graphical editor for such curves
should be a preferable approach to the use of these objects in a
\IXpov{} scene.

The executable file of this program, if installed on your system,
will probably probably have the file name \IXpkg{}, which
might be remembered as `edit povray spline'. On some systems the
file might require a dot-suffix, e.\ g.\ `\dtythispkgname.exe'. (The
suffix might not be necessary for command invocation.) In this document
the program might be referred to variously as ``\dtythispkgname'',
``the program'', ``the editor'', or ``the spline editor''.

The program was begun in late 1999 as a typical `scratch one's own itch'
project with no thought of development beyond addressing a specific
need at the time. It soon became apparent that a little more
development was warranted, and soon after that that it might be
useful to others if developed sufficiently for public distribution.
Unfortunately, circumstances arrived which prevented further attention
to this program (and most other spare-time endeavors). Only recently
(early 2005) has the program regained the attention it had seemed to
deserve.


\section{At The Time of This Writing}%WX%3||||

It is mid-March 2005. The spline editor will soon be ready for
an initial public release on the Internet. The first release will
certainly be in the alpha development stage, but it is already
useful (to the author in any case), and has not been observed to
crash, or to corrupt files or the system, to cause large scale
malfunctions of the electrical power delivery system,
or any such catastrophe.


\subsection{On the Hope That the Program is Useful}%WX%4||402000||

Distribution will be done only for the possibility that the program
might prove useful. The author makes no claim that the program will
suit any purpose, and does not claim that the program will perform
properly or that it will not cause damage to your system, your life,
or the world at large. Proper legal disclaimers may be found in
the software license that is include in the distribution package.
Needles to say (the author hopes), the software would not have been
distributed if catastrophic failure seemed likely.

One motivation for public distribution is that interested persons ---
users --- will return suggestions and even code patches as a
contribution to the further improvement of the program. Please note
that stating this hope does not constitute a commitment to continued
development or maintenance of the program.

